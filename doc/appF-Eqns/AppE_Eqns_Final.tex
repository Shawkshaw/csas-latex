% AppF-EqnsAlone.tex - equations appendix as a stand-alone file.
%  Now stripping out formatting to be a proper appendix. 2/10/12.
% AppF-Eqns.tex - POP 3CD equations. Using final YMR as template. 
%  First making edits needed for both 3CD and 5DE, then save, copy
%  to 5DE, and continue for ones just needed for 3CD. 
%  Not worrying about formatting for now. 1st October 2012. 
% YMRrev-AppF-EqnsLeftJust.text - left justifying. 19th Sept 2012.
% YMRrev-AppF-Eqns.tex - revising YMR for Res Doc. Not using .sty
%  file as not complete yet, and don't want to make 11pt text as 
%  would have to fiddle with equation tables. So format should be 
%  same as for POP. Table captions not indented as they're not real
%  captions. Trying Jaclyn's citation fix for that: \leftskip
%  29th March 2012.
% YMR-AppF-Eqns.tex - equations for YMR appendix. From revised POP ones as
%  of 15th May 2011.

% Style, fonts, margins:
  % \documentstyle[12pt,ifthen,fleqn]{article}
  \documentclass[11pt]{article}   % fleqn ,ifthen
  %\usepackage{ResDocSty}  % cannot call due to chapter conflicts
  \usepackage{nccmath}     % from web for fleqn
  \usepackage{natbib}     % need for bibliography ?
  \usepackage{lineno}
  \usepackage[top=1in, bottom=1in, left=1in, right=1in]{geometry} 
  %\usepackage{geometry}
  % \linenumbers

%\bibliographystyle{ecology} % didn't work

\raggedright
\usepackage{fancyhdr}
\pagestyle{fancy}
% \fancyfoot{}
% \fancyfoot[LE,LO]{DRAFT -- Non-citable working paper}{WP 2011/**}
%\lhead{DRAFT -- Non-citable working paper}
%\rhead{CSAP WP 2013/P27}
\lhead{}
\rhead{}
\lfoot{Appendix E -- Model Equations}
\rfoot{Silvergray Rockfish} % WP 2010/035}
\renewcommand{\footrulewidth}{0.4pt}
\renewcommand{\headrulewidth}{0pt}



% These were Jon's:
%   \oddsidemargin 0in               \evensidemargin 0in
%   \textwidth 6.3in                 \textheight 8.9in
%   \topmargin -0.9in
  \parskip 2ex   % was 3ex
%  \parindent 2em
%  \def\mybaselineskip{3.8ex}   % for end of tables. Though changes text as well. Maybe just need once at start? Seems to occur three times with no other value anywhere. Maybe just go with usual:
% \renewcommand{\baselinestretch}{1.9} % Prob don't need.
\setlength\parindent{0pt}

% From levyfleet Ecology manuscript:
\geometry{verbose,tmargin=2.4cm,bmargin=2.4cm,lmargin=2.6cm,rmargin=2.6cm} % For ecology margin widths, they had a4paper after verbose, I tried letter, didn't work so deleting. all were 2.4, but for A4, so setting here to 2.6 left and right


  \def\AppLet{E}                   % Appendix letter
  \def\StartP{109}                 % page start
  \def\finalYr{2014}               % final year in model
  \def\gcomm{7}                    % index g for commercial fishery (if more than one, this will need to be revised)
 %\def\schematic{1~}               % Figure number for schematic fig
                                   %  (in main text)
  \font\mbf=cmmib10 scaled 1200    % computer modern math italic bold
  \font\sbf=cmbsy10 scaled 1200    % computer modern symbol bold
  \def\bfmi#1{{\hbox{\mbf #1}}}    % bold face math italic macro
  \def\bfms#1{{\hbox{\sbf #1}}}    % bold face math symbol macro
  \def\bfTh{{\bf \Theta}}          % bold Theta
  \def\bfPh{{\bf \Phi}}            % bold Phi
  \def\bfleq{\,\bfms{\char'24}\,}  % bold <= (less than or equal)
  \def\bfgeq{\,\bfms{\char'25}\,}  % bold >= (greater than or equal)
  \def\bfeq{\mbox{\bf\,=\,}}       % bold =
  \def\bft{\bfmi{t}}               % bold t
  \def\bfT{\bfmi{T}}               % bold T   - need for headings
  \def\rbT{\mbox{\bf T}}           % Roman bold T
  \def\rbU{\mbox{\bf U}}           % Roman bold U
  \def\winf{w_\infty}

  \def\veq{\vspace{-4ex}} % contraction around equations
  \def\vec{\vspace{-3ex}} % contraction around centering
%\def\headc{\vspace{-2ex}} % contraction after 'fake' subsubheading
\def\headc{\vspace{-1ex}} % contraction after 'fake' subsubheading
\def\subsub#1{\noindent {\bf #1} \headc}    % fake subheading

\newcommand{\popQCS}{Edwards-etal:2012pop}
\newcommand{\ymr}{Edwards-etal:2012ymr}
\newcommand{\area}{3CD}
\newcommand{\inarea}{coastwide}
\newcommand{\other}{5DE}
\newcommand{\popWCVI}{Edwards-etal:20133CD}
\newcommand{\popWCHG}{Edwards-etal:20135DE}
\newcommand{\EstM}{`Estimate \emph{M}'}
\newcommand{\FixM}{`Fix \emph{M}'}
\newcommand{\Bmsy}{$B_\mathrm{MSY}$}
\newcommand{\umsy}{$u_\mathrm{MSY}$}


%   \def\beq{\veq \begin{fleqn}}   % Should do flush, rather than for
%   \def\eeq{\end{fleqn} \veq}     %  whole document
% -6ex is better but overlaps, but 4 seems to give too much space. 
%  Fix when put into Sweave document.
  \def\beq{\vspace{-5ex} \begin{fleqn} \begin{equation}}   % Should do flush, rather than for
  \def\eeq{\end{equation} \end{fleqn} \vspace{-5ex}}     %  whole document
\def\tabline{\vspace{2ex} \hrule \vspace{2ex}}


% Set default numbering:
  %\setcounter{secnumdepth}{3}    % To number subsubheadings-ish
  %\setcounter{chapter}{5}  % temporary for standalone chapters (3 if not using \thechapter in chapter{}, 4 if using it... who knows why
  %\renewcommand{\thechapter}{\Alph{chapter}} % ditto
  \renewcommand{\thesection}{\AppLet.\arabic{section}}   
  \renewcommand{\theequation}{\AppLet.\arabic{equation}}
  \renewcommand{\thefigure}{\AppLet.\arabic{figure}}
  \renewcommand{\thetable}{\AppLet.\arabic{table}}
%  \renewcommand{\thepage}{\AppLet\arabic{page}}

  %\renewcommand{\thesubsection}{\AppLet.\arabic{section}.\arabic{subsection}} % not used?
% Tables not automatically done.
\renewcommand{\rmdefault}{phv}   % Arial
\renewcommand{\sfdefault}{phv}   % Arial

% Andy deleting a lot of Jon's that we aren't now using. See POPawateaeqns.tex

\def\vsd{\vspace*{1ex}}     % Aha - there's a whole bunch of these plus -ve
\def\newp{\vfill \break}

\def\Var{\mbox{Var}}             \def\Cov{\mbox{Cov}}

% Equation reference: #1=internal label saved in AUX file
  \newcommand{\eref}[1]{(\ref{#1})}

% Andy's usual:
\newcommand{\eb}{\begin{eqnarray}}
\newcommand{\ee}{\end{eqnarray}}

\renewcommand{\bibname}{References}

% ****************** end macros ********************

\begin{document}   
\setcounter{page}{\StartP}
% \baselineskip \mybaselineskip

\begin{center}
%\noindent {\bf \ APPENDIX \AppLet. DESCRIPTION OF CATCH-AT-AGE MODEL}
\noindent {\bf \large APPENDIX~\AppLet.~~MODEL EQUATIONS}
\end{center}

%See notes and ** for things to check with Paul.

% \setcounter{section}{0}

% \subsection{Model outline and assumptions}
% {\bf MODEL OUTLINE AND ASSUMPTIONS}
{\bf INTRODUCTION}

% Given the rich catch-at-age data available for Pacific ocean perch, we were able to use a sex-specific, age-structured model in a Bayesian framework. In particular, we could simultaneously estimate the steepness of the stock-recruit function and sex-specific natural mortality parameters. 

We used a sex-specific, age-structured model in a Bayesian framework. In particular, the model can simultaneously estimate the steepness of the stock-recruitment function and separate mortalities for males and females. This approach follows that used in our recent stock assessments of Pacific Ocean Perch (POP) in Queen Charlotte Sound,  west coast Vancouver Island, and west coast Haida Gwaii \citep{\popQCS, \popWCVI, \popWCHG} and Yellowmouth Rockfish along the Pacific coast of Canada \citep{\ymr}. 

The model structure is the same as that used previously, and, as for the Yellowmouth Rockfish assessment and the two recent POP assessments, we used the new weighting scheme of \citet{fran11} desribed below.
% The same methodology is used in the companion POP assessments for areas 3CD \citep{\popWCVI} and 5DE \citep{\popWCHG}, with the differences being in the data (e.g., survey series and the available age data).

Implementation was done using a modified version of the Coleraine statistical catch-at-age software \citep{hmpeps03} called Awatea (A.~Hicks, NOAA, pers.~comm.). Awatea is a platform for implementing the AD (Automatic Differentiation) Model Builder software \citep{ADMB2009}, which provides (a)~maximum posterior density estimates using a function minimiser and automatic differentiation, and (b)~an approximation of the posterior distribution of the parameters using the Markov Chain Monte Carlo (MCMC) method, specifically using the Hastings-Metropolis algorithm \citep{gelman2004bayesian}. 

% Not all of these refs may be in my .bib file:

Running of Awatea was streamlined using code we wrote in R \citep{R2013}, rather than the original Excel implementation. Figures and tables of output were automatically produced through R using code adapted from the R packages {\tt scape} \citep{magn09} and {\tt scapeMCMC} \citep{ms07}. We used the R software {\tt Sweave} \citep{leis02} to automatically collate, via \LaTeX, the large amount of figures and tables into a single pdf file for each model run.

Below we describe details of the age-structured model, the Bayesian procedure, the reweighting scheme, the prior distributions, and the methods for calculating reference points and performing projections.


% Figure \schematic**\marginpar{AME} in the main text shows the main inputs and outputs. 

{\bf MODEL ASSUMPTIONS}

The assumptions of the model are:

\noindent 1.~The stock \inarea{} was treated as a single stock.

\noindent 2.~Catches were taken by a single fishery, known without error, and occurred in the middle of the year.

\noindent 3.~A time-invariant Beverton-Holt stock-recruitment relationship was assumed, with log-normal error structure.

\noindent 4.~Selectivity was different between sexes and surveys and invariant over time. Selectivity parameters were estimated when ageing data were available.

\noindent 5.~Natural mortality was held invariant over time, and estimated independently for females and males.

\noindent 6.~Growth parameters were fixed and assumed to be invariant over time.

\noindent 7.~Maturity-at-age parameters for females were fixed and assumed to be invariant over time. Male maturity did not need to be considered, because it was assumed that there were always sufficient mature males.

\noindent 8.~Recruitment at age 1 was 50\% females and 50\% males.

\noindent 9.~Fish ages determined using the surface ageing methods (before 1978) were too biased to use \citep{beam79}. Ages determined using the otolith break-and-burn methodology \citep{macl97} were aged without error. 

\noindent 10.~Commercial samples of catch-at-age in a given year were assumed to be representative of the fishery if there were $\geq$4 samples.

\noindent 11.~Relative abundance indices were assumed to be proportional to the vulnerable biomass at the mid point of the year, after half of the catch and half of the natural mortality had been accounted for.

\noindent 12.~The age composition samples were assumed to come from the middle of the year after half of the catch and half of the natural mortality had been accounted for.

{ \bf MODEL NOTATION AND EQUATIONS}

The notation for the model is given in Table \AppLet.1, the model equations in Tables \AppLet.2 and \AppLet.3, and description of prior distributions for estimated parameters in Table \AppLet.4. The model description is divided into the deterministic components, stochastic components and Bayesian priors. Full details of notation and equations are given after the tables. % (in the order of appearance in the tables, as far as is practical). 

% Discussion of the Bayesian aspects is then given in Section \ref{sec:Bayes}.

The main structure is that the deterministic components in Table \AppLet.2 can iteratively calculate numbers of fish in each age class (and of each sex) through time. The only requirements are the commercial catch data, weight-at-age and maturity data, and known fixed values for all parameters.

Given we do not have known fixed values for all parameters, we need to estimate many of them, and add stochasticity to recruitment. This is accomplished by the stochastic components given in Table \AppLet.3. 

Incorporation of the prior distributions for estimated parameters gives the full Bayesian implementation, the goal of which is to minimise the objective function $f(\bfTh)$ given by \eref{objfn}. This function is derived from the deterministic, stochastic and prior components of the model. % , which are now described in turn after the tables.



% Note that notation is consistent within this Appendix (and results section**), though not with appendices that .....[**see what I wrote on Rowan's].

\clearpage

% ********************** Table 1 ************************************


% \baselineskip 2.5ex \vspace{-2ex}

\noindent \begin{tabular}{ll} 
\multicolumn{2}{l} {\it {Table \AppLet.1 (continued overleaf). Notation for the catch-at-age model.}} \\ 
 & \\
\hline
{\bf Symbol} & \multicolumn{1}{c}{{\bf Description and units}} \\ \hline \ \\[-.5ex]
%
& \multicolumn{1}{c}{\bf{Indices (all subscripts)}} \\[0.5ex]
$a$ & age class, where $a = 1, 2, 3, ... A$, and $A = 32$ is the accumulator age class\\
$t$ & model year, where $t = 1, 2, 3, ... T$, corresponds to actual years 1940, 1941,\\
 & ~~1942, ..., \finalYr, and $t=0$ represents unfished equilibrium conditions\\
$g$ & index for certain data:\\
%  $g = 1,2,3$ are survey series 1, 2, 3, and $g=6$ is commercial\\
 & ~~1 - West Coast Haida Gwaii synoptic survey series\\
 & ~~2 - Hecate Strait synoptic survey series\\
 & ~~3 - Queen Charlotte Sound syoptic survey series\\
 & ~~4 - West Coast Vancouver Island synoptic survey series\\
 & ~~5 - GB Reed historical survey series\\
 & ~~6 - National Marine Fisheries Service Triennial survery series\\
 & ~~\gcomm~ - commercial trawl data\\
$s$ & sex, $1 =$ females, $2 =$ males\\
\\[-.5ex]
 & \multicolumn{1}{c}{\bf{Index ranges}} \\
$A$ & accumulator age-class, $A=32$ \\
$T$ & number of model years, $T = 75$\\
${\bf T}_g$ & sets of model years for survey abundance indices from series $g$, $g=1,...,6$, listed here\\
 & ~~~~ for clarity as actual years (subtract 1939 to give model year $t$):\\
 & ~~${\bf T}_1 =$ \{1997, 2006, 2007, 2008, 2010, 2012\}\\
 & ~~${\bf T}_2 =$ \{2005, 2007, 2009, 2011, 2013\}\\
 & ~~${\bf T}_3 =$ \{2003, 2004, 2005, 2007, 2009, 2011, 2013\}\\
 & ~~${\bf T}_4 =$ \{2004, 2006, 2008, 2010, 2012\}\\
 & ~~${\bf T}_5 =$ \{1967, 1969, 1971, 1973, 1976, 1977, 1984\}\\
 & ~~${\bf T}_6 =$ \{1980, 1983, 1989, 1992, 1995, 1998, 2001\}\\
${\bf U}_g$ & sets of model years with proportion-at-age data, $g=1,...,4$ (listed here as actual years):\\
% & ~~~~actual years):\\
 & ~~${\bf U}_1 =$ \{1997, 2010\}\\
 & ~~${\bf U}_2 =$ \{2009, 2011\}\\
 & ~~${\bf U}_3 =$ \{2004, 2009, 2011\}\\
 & ~~${\bf U}_4 =$ \{2004, 2010\}\\
 & ~~${\bf U}_\gcomm =$ \{1979,...,1981, 1983, 1990,...,2007, 2009,...,2011\}\\
\\[-.5ex]

& \multicolumn{1}{c}{{\bf Data and fixed parameters}} \\[0.5ex]
$p_{atgs}$ & observed weighted proportion of fish from series $g$ in each year $t \in {\bf U}_g$ that are\\
 & ~~  age-class $a$ and sex $s$; so $\Sigma_{a=1}^{A} \Sigma_{s=1}^2 p_{atgs} = 1$ for each $t  \in {\bf U}_g$, $g=1,...,4$\\
% $\tau_{tg}$ & inverse of assumed sample size that yields corresponding $p_{atgs}$\\
$n_{tg}$ & assumed sample size that yields corresponding $p_{atgs}$\\
$C_t$ & observed catch biomass in year $t = 1, 2, ..., T-1$, tonnes\\
$w_{as}$ & average weight of individual of age-class $a$ of sex $s$ from fixed parameters, kg\\ 
$m_a$ & proportion of age-class $a$ females that are mature, fixed from data\\
$I_{tg}$ & biomass estimates from surveys $g = 1,...,6$, for year $t \in {\bf T}_g$, tonnes\\
$\kappa_{tg}$ & standard deviation of $I_{tg}$\\
% **$f_{1t}, f_{2t}$ & *needed?, what's the assumption? fraction of catch taken prior to research and charter vessel surveys  \\
$\sigma_R$ & standard deviation parameter for recruitment process error, $\sigma_R = 0.6$ \\
% $v_{gR}$ & variance parameter for right limb of selectivity curve for series $g = 1,...,7$; $v_{gR} = 100$\\
% $v_{R}$ & variance parameter for right limb of selectivity curves, $v_{R} = e^{100}$\\
% & ~~fixed at 100 to give no descending limb
\\[-.5ex]
%
%
\end{tabular} \newp % \baselineskip \mybaselineskip


\noindent \begin{tabular}{ll} 
\multicolumn{2}{l} {{\it Table \AppLet.1 (cont.). Notation for the catch-at-age model.}} \\ 
 & \\
\hline
{\bf Symbol} & \multicolumn{1}{c}{{\bf Description, with fixed values and/or units where appropriate}} \\ \hline \ \\[-.5ex]
%
& \multicolumn{1}{c}{{\bf Estimated parameters}} \\[0.5ex]
$\bfTh$ & set of estimated parameters\\
% R_0, M_1, M_2, h, q_1, q_2, q_3, q_4, q_5, \mu_1, \mu_2, \mu_6, \Delta_1, \Delta_2, \Delta_6, v_{1L}, v_{2L}, v_{6L}

$R_0$ & virgin recruitment of age-1 fish (numbers of fish, 1000s)\\
$M_{s}$ & natural mortality rate for sex $s$, $s=1,2$\\
$h$ & steepness parameter for Beverton-Holt recruitment\\
$q_g$ & catchability for survey series $g = 1,...,6$\\ 
$\mu_g$ & age of full selectivity for females for series $g = 1,...,\gcomm$\\
$\Delta_g$ & shift in vulnerability for males for series $g = 1,...,\gcomm$\\
$v_{gL}$ & variance parameter for left limb of selectivity curve for series $g = 1,...,\gcomm$\\
$s_{ags}$ & selectivity for age-class $a$, series $g = 1,...,\gcomm$, and sex $s$, calculated from\\
 & ~~the parameters $\mu_g, \Delta_g$ and $v_{gL}$\\ 
            % and $v_{gR}$ \\

% $v_{gL}, v_{gR}$ & variance parameters for left and right limbs of selectivity for series $g = 1,...,7$; $v_{gR}= 100$ fixed to give no descending limb\\

% $\sigma_I$ & standard deviation parameter for initial age structure error \\
% **$\tau_2$ & standard deviation of age proportion measurement error \\
% **$\kappa^2$ & combined variance $\sigma_1^2 + \tau_1^2$ \\
% **$\rho$ & variance ratio $\sigma_1^2 / \kappa^2$, fixed in the model
%  analysis \\

$\alpha$, $\beta$ & alternative formulation of recruitment: $\alpha = (1 - h) B_0 / (4 h R_0)$ and \\
 & ~~$\beta = (5 h - 1) / 4 h R_0$\\ 
$\widehat{x}$ & estimated value of observed data $x$\\
% $\widehat{\cdot}$ & estimated value of observed data represented by $\cdot$
\\[-.5ex]

& \multicolumn{1}{c}{{\bf Derived states}} \\[0.5ex]
$N_{ats}$ & number of age-class $a$ fish of sex $s$ at the start of year $t$, 1000s\\
$u_{ats}$ & proportion of age-class $a$ and sex $s$ fish in year $t$ that are caught\\
$u_t$ & ratio of total catch to vulnerable biomass in the middle of the year\\% (exploitation rate of 'fully selected' fish in year $t$\\
 & ~~(exploitation rate)\\
% **$R_t$ & age-class 1 recruitment in year $t$, $10^6$ \\
% **$P_t$ & exploitable population numbers at the start of year $t$, $10^6$ \\
% **$B_t$ & exploitable population biomass at the start of year $t$, kt \\
$B_t$ & spawning biomass (mature females) at the start of year $t$,\\
& ~~$t=1,2,3,...,T$; tonnes \\
% & **rocksole/can different to manual, presumably Awatea thing,and MCMC\$B is this $B_t$ \\
$B_0$ & virgin spawning biomass (mature females) at the start of year $0$, tonnes \\
$R_t$ & recruitment of age-1 fish in year $t$, $t=1,2,...,T-1$, numbers of fish, 1000s\\
$V_t$ & vulnerable biomass (males and females) in the middle of year $t$,\\
      & ~~$t=1,2,3,...,T$; tonnes\\

\\[-.5ex]
& \multicolumn{1}{c}{{\bf Deviations and likelihood components}} \\[0.5ex]
$\epsilon_t$ & Recruitment deviations arising from process error\\
$\log L_1(\bfTh | \{ \epsilon_t \}) $ & log-likelihood component related to recruitment residuals\\
$\log L_2(\bfTh | \{ \widehat{p}_{atgs} \} )$ & log-likelihood component related to estimated proportions-at-age\\
$\log L_3(\bfTh | \{ \widehat{I}_{tg} \} )$ &  log-likelihood component related to estimated survey biomass indices \\
$\log L(\bfTh)$ & total log-likelihood \\
\\[-.5ex]

& \multicolumn{1}{c}{{\bf Prior distributions and objective function}} \\[0.5ex]
$\pi_j(\bfTh)$ & Prior distribution for parameter $j$ \\
$\pi(\bfTh)$ & Joint prior distribution for all estimated parameters\\
$f(\bfTh)$ & Objective function to be minimised\\
\end{tabular} \newp % \baselineskip \mybaselineskip




% ********************** Table 2 ************************************

\def\bec{ \begin{center} \hspace{-15ex}}
\def\eec{\end{center} \vspace{-1ex}}

\leftskip=1.5em	   %indents caption (that isn't done as a caption)
\parindent=-1.5em  % then revert back afterwards

{\it Table \AppLet.2. Deterministic components (continued overleaf). Using the catch, weight-at-age and maturity data, with fixed values for all parameters, the initial conditions are calculated from \eref{de1}-\eref{dS0}, and then state dynamics are iteratively calculated through time using the main equations \eref{df1}-\eref{df3}, selectivity functions \eref{self} and \eref{selm}, and the derived states \eref{dSt}-\eref{uats}. Estimated observations for survey biomass indices and proportions-at-age can then be calculated using \eref{dg1} and \eref{dg3}. In Table \AppLet.3, the estimated observations of these are compared to data.}% \tabline 

\leftskip=0em
\parindent=-0em


\vspace{1ex} \hrule

% Andy taking out Data and Fixed parameters section - see POPawateaeqns.1tex.

\vspace{-1ex}

\bec {\bf State dynamics
  (${\bf 2 \bfleq \bft \bfleq \bfT, \, \bfmi{s} \bfeq 1,2}$\,)} \eec

\beq N_{1ts} = 0.5 R_t  \label{df1} \eeq \vsd \vsd \vsd    % was R_{t-1}

\beq N_{ats} = e^{-M_s} ( 1 - u_{a-1,t-1, s} ) N_{a-1,t-1,s}
  \,; \ \ \ 2 \leq a \leq A-1
  \label{df2} \eeq \vsd \vsd \vsd

\beq  N_{Ats} = e^{-M_s} ( 1 - u_{A-1,t-1, s}) N_{A-1,t-1,s}
  + e^{-M_s} ( 1 - u_{A,t-1, s} ) N_{A,t-1,s} 
%  N_{At} = e^{-M} \left[ N_{A-1,t-1} + N_{A,t-1}
%  - (u_{A-1,t-1} + u_{A,t-1}) C_{t-1} \right]} 
\label{df3}
  \eeq \vsd

\bec {\bf Initial conditions ($\bf \bft \bfeq 1$)} \eec  % , {\bf s} \bfeq 1,2$)}

\beq N_{a1s} = 0.5 R_0 e^{-M_{s}(a-1)} \, ; \ \ \
  1 \leq a \leq A -1, \,  s = 1,2  \label{de1} \eeq \vsd \vsd \vsd \vsd

\beq N_{A1s} = 0.5 R_0\, \frac{e^{-M_{s}(A-1)}}{1-e^{-M_{s}}} \, ; \ \ \  s = 1,2
  \label{de2} \eeq \vsd \vsd \vsd \vsd

\beq B_0 = B_1 = \sum_{a=1}^A w_{a1} m_a N_{a11} 
%  & = & 0.5 R_0\, \sum_{a=1}^A w_{a2} m_a N_{a12}\\     \frac{e^{-M(A-1)}}{1-e^{-M}} Was going to write out, but no need._s
  \label{dS0} \eeq \vsd


% \bec {\bf Selectivity ($\bf 1 \bfleq {\bf g} \bfleq 4$)} \eec
\bec {\bf Selectivities ($\bfmi{g} \bf = 1,...,\gcomm$)} \eec 

\vspace{-2ex}

\begin{fleqn}     % \beq puts -ve vspace, not good for { ...
\begin{equation}
s_{ag1} = \left\{
 \begin{array}{ll}
 e^{-(a - \mu_g)^2 / v_{gL}}, & a \leq \mu_g\\
 1, & a > \mu_g
% e^{-(a - \mu_g)^2 / v_{R}}, & a > \mu_g
\label{self}
 \end{array}
\right.
\end{equation}
\end{fleqn}
\vspace{-5ex}

\begin{fleqn}
\begin{equation}
s_{ag2} = \left\{
 \begin{array}{ll}
 e^{-(a - \mu_g - \Delta_g)^2 / v_{gL}}, & a \leq \mu_g + \Delta_g\\
 1, & a > \mu_g + \Delta_g
% e^{-(a - \mu_g - \Delta_g)^2 / v_{R}}, & a > \mu_g + \Delta_g
\label{selm}
 \end{array}
\right.
\end{equation}
\end{fleqn}

\noindent \hrule %\tabline
\newp      % this [\vfill \break] forces everything to be on this page. 



{\it \noindent {Table \AppLet.2 (cont.)}} \tabline

\bec {\bf Derived states ($\bf 1 \bfleq \bft \bfleq \bfT - 1$\,)} \eec

\beq B_t = \sum_{a=1}^A w_{a1} m_a N_{at1}
  \label{dSt} \eeq \vsd \vsd \vsd \vsd \vsd

\beq R_t = \frac{4 h R_0 B_{t-1}}{(1-h) B_0 + (5 h - 1) B_{t-1}} ~~\left( \equiv  \frac{B_{t-1}}{\alpha + \beta B_{t-1}} \right)
 \label{Rt}
  \eeq \vsd \vsd \vsd \vsd \vsd

\beq V_t = \sum_{s=1}^2 \sum_{a=1}^A e^{-M_{s}/2}\, w_{as} \, s_{a\gcomm s} \, N_{ats}
\label{Vt}
  \eeq \vsd \vsd \vsd \vsd \vsd


\beq u_{t} = \frac{C_t}{V_t}   % \,; \ \ \ 
  \label{ut} \eeq \vsd \vsd \vsd \vsd \vsd

\beq u_{ats} = s_{a\gcomm s} \, u_t \,; \ \ \ 1 \leq a \leq A, ~s=1,2  
  \label{uats} \eeq \vsd


% \bec {\bf Recruitment
%  ($\bf 1 \bfleq \bft \bfleq \bfT - 2$\,)} \eec


\bec {\bf Estimated observations} \eec
%  ($\bf 1 \bfleq \bft \bfleq \bfT$\,)} \eec

\beq \widehat{I}_{tg} = q_g  \sum_{s=1}^2 \sum_{a=1}^A e^{-M_{s}/2} (1 - u_{ats}/2)  w_{as} s_{ags} N_{ats} \,; \ \ \
  t \in {\bf T}_g, ~g = 1,2,3  \label{dg1}  \eeq \vsd \vsd \vsd \vsd \vsd

\beq \widehat{p}_{atgs} = \frac{e^{-M_{s}/2} (1 - u_{ats}/2) s_{ags} N_{ats}}{\sum_{s=1}^2 \sum_{a=1}^A e^{-M_{s}/2} (1 - u_{ats}/2) s_{ags} N_{ats}}; \ \ \ 1 \leq a \leq A,~ t \in {\bf U}_g,~g=1,4,~s=1,2  \label{dg3} \eeq \vsd 


\vsd
\vsd
\vsd
\vsd
\vsd

\noindent \hrule %\tabline
\newp

% ********************** Table 3 ************************************

\leftskip=1.5em	   %indents caption (that isn't done as a caption)
\parindent=-1.5em  % then revert back afterwards
{\it Table \AppLet.3. Calculation of likelihood function $L(\bfTh)$ for stochastic components of the model in Table \AppLet.2, and resulting objective function $f(\bfTh)$ to be minimised.} \tabline%
\leftskip=0em
\parindent=-0em

% \seteq{L}

\bec {\bf Estimated parameters} \eec \vspace{1ex}

% \beq \bfTh = \left( { \vrule height 2.5ex width 0ex} \{\mu_g\}, \{v_{gL}\}, \{\Delta_g\}, \{q_g\}, \{M_s\}, R_0, h \right)
% \beq \bfTh = \left\{ \mu_1, \mu_2, \mu_6, v_{1L}, v_{2L}, v_{6L}, \Delta_1, \Delta_2, \Delta_6, q_1, q_2, q_3, M_1, M_2, R_0, h \right\}
% redoing in order that output is in:
%\beq \bfTh = \left\{ R_0, M_1, M_2, h, q_1, q_2, q_3, \mu_1, \mu_4, \Delta_1, \Delta_4, v_{1L}, v_{4L} \right\} % POP ???
\beq \bfTh = \left\{ R_0; M_{1,2}; h; q_{1,...,6}; \mu_{1,...4,\gcomm}; \Delta_{1,...,4,\gcomm}; v_{1,...4,\gcomm L} \right\} % SGR 2013
  \label{lpar} \eeq \vsd

% \beq \kappa^2 = \sigma_1^2 + \tau_1^2 \, , \ \
%  \rho = \frac{ \sigma_1^2}{\sigma_1^2 + \tau_1^2}  \label{lkr} \eeq \vsd

% \beq \sigma_1^2 = \rho \kappa^2 \, , \ \
%  \tau_1^2 = (1-\rho)\kappa^2  \label{lst} \eeq \vsd


\bec {\bf Recruitment deviations} \eec \vspace{1ex}

\beq \epsilon_t = \log R_t  - \log B_{t-1} + \log(\alpha + \beta B_{t-1}) + \sigma_R^2/2 \, ; ~~ 1 \leq t \leq T-1 \label{epst} \eeq \vsd

% Two equations for if there was error in initial age structure _s
% \beq \xi_{as} = \log N_{a1s} - \log R_0 + \log 2 + M(a - 1) + \sigma_I^2  \, ;~~1 \leq a \leq A-1, s = 1,2 \label{xias} \eeq \vsd

% \beq \xi_{As} = \log N_{A1s} - \log R_0 + \log 2 + M(A - 1) + \log (1 - e^{- M}) + \sigma_I^2  \, ; s = 1,2  \label{xiAs}  \eeq \vsd


\bec {\bf Log-likelihood functions} \eec \vspace{1ex}

\beq \log L_1(\bfTh | \{ \epsilon_t \}) =
  - \frac{T}{2} \log 2 \pi - T \log \sigma_R - \frac{1}{2 \sigma_R^2} \sum_{t=1}^{T-1} \epsilon_t^2
  \label{ll1} \eeq \vsd \vsd \vsd \vsd \vsd
%  \sqrt{1-\gamma^2} \, ({\sqrt{2\pi}\,\sigma_1})^{2-A-T} \,
%  \exp\left[ - \frac{1}{2\sigma_1^2} \left( (1-\gamma^2) \xi_{2-A}^2
%    + \sum_{t=3-A}^T \xi_t^2 \right) \right]

% If there was error in initial age structure
%\beq \log L_2(\bfTh | \{ \xi_{as} \} ) =
%  - A \log 2 \pi - 2 A \log \sigma_I - \frac{1}{2 \sigma_I^2} \sum_{s=1}^{s=2} \sum_{a=1}^A \xi_{as}^2
%  \label{ll2} \eeq \vsd \vsd

\vspace{-5ex}
\begin{fleqn}
\begin{eqnarray}
\nonumber \log L_2(\bfTh | \{ \widehat{p}_{atgs} \} ) = - \frac{1}{2} \sum_{g=1,4} \sum_{a=1}^A \sum_{t \in {\bf U}_g} \sum_{s=1}^2 \log \left[ p_{atgs} ( 1 - p_{atgs} ) + \frac{1}{10 A} \right]~~~~~~~~~~~~~~~~~~~~~~ \\
 + \sum_{g=1,4} \sum_{a=1}^A \sum_{t \in {\bf U}_g} \sum_{s=1}^2 \log \left[ \exp \left\{ \frac{- ( p_{atgs} - \widehat{p}_{atgs} )^2~n_{tg}}{2 \left( p_{atgs} ( 1 - p_{atgs} ) + \frac{1}{10 A} \right) } \right\} + \frac{1}{100} \right]
\label{ll2} % \eeq
\end{eqnarray}
\end{fleqn}
\vspace{-5ex}
 \vsd \vsd

\vsd
\vsd
\vsd

\beq 
\log L_3(\bfTh | \{ \widehat{I}_{tg} \} ) = \sum_{g=1}^3 \sum_{t \in {\bf T}_g} \left[ - \frac{1}{2} \log 2 \pi - \log \kappa_{tg} - \frac{ ( \log I_{tg} - \log \widehat{I}_{tg} )^2 }{2 \kappa_{tg}^2 }  \right]
% \log L_2(\bfTh | \{ \xi_{as} \} ) =
  \label{ll3} \eeq \vsd \vsd \vsd \vsd \vsd

\beq \log L(\bfTh) = \sum_{i=1}^3 \log L_i(\bfTh | \cdot)  \label{llk} \eeq \vsd

\bec {\bf Joint prior distribution and objective function} \eec \vspace{1ex}

\beq \log(\pi(\bfTh)) = \sum_{j} \log(\pi_j (\bfTh)) 
\label{jointprior}\eeq

\vsd \vsd \vsd \vsd

\beq f(\bfTh) = - \log L(\bfTh) - \log(\pi(\bfTh)) 
\label{objfn} \eeq


\vsd
\vsd
\vsd
\vsd

\noindent \hrule %\tabline

\clearpage

\begin{table}[tp]
% {\it Table \AppLet.4. Prior distributions for estimated parameters and lower and upper bounds within which parameter estimates are constrained. Uniform priors are uniform between the bounds, normal$(x,y)$ means normal distribution with mean $x$ and standard deviation $y$, and similarly for lognormal and beta distributions. The resulting probability density functions are the $\pi_j(\bfTh)$ functions that contribute to the joint prior distribution in \eref{jointprior}.}
\leftskip=1.5em
\parindent=-1.5em
{\it Table \AppLet.4. Details for estimation of parameters, including prior distributions with corresponding means and standard deviations, bounds between which parameters are constrained, and initial values to start the minimisation procedure for the MPD (mode of the posterior density) calculations. For uniform prior distributions, the bounds completely parameterise the prior. The resulting non-uniform prior probability density functions are the $\pi_j(\bfTh)$ functions that contribute to the joint prior distribution in \eref{jointprior}.}
%\vsd \\
\leftskip=0em
\parindent=-0em
\begin{center}
\begin{tabular}{lcccc} 
\hline \\[-1.5ex]
Parameter & Prior         & Mean, standard & Bounds & Initial\\ 
          &  distribution &  deviation     &  & value\\ 
\hline %\\[-.5ex]
% copy table from Sweave tex file
$R_0$ & uniform & -- & [1,~100,000] & 3,000\\
$M_{1}, M_{2}$ & normal & 0.06, 0.006 & [0.01, 0.12] & 0.06\\
$h$ & beta & 4.574, 2.212 & [0.2, 0.999] & 0.674\\
$\log q_{1,...,6}$ & uniform & 0, 0.6 & [-5, 5] & 0\\
$\mu_{1,2}$ & normal & 10.8, 3.24 & [5, 32] & 10.8\\
$\mu_{3}$ & normal & 13.3, 4.0 & [5, 32] & 13.3\\
$\mu_{4}$ & normal & 15.4, 4.62 & [5, 32] & 15.4\\
$\mu_{\gcomm}$ & uniform & 10.5, 3.15 & [5, 32] & 10.5\\
$\log v_{1,2L}$ & normal & 2.08, 0.62 & [-15, 15] & 2.08\\
$\log v_{3L}$ & normal & 3.3, 1.0 & [-15, 15] & 3.3\\
$\log v_{4L}$ & normal & 3.44, 1.03 & [-15, 15] & 3.44\\
$\log v_{\gcomm L}$ & uniform & 1.52, 0.46 & [-15, 15] & 1.52\\
$\Delta_{1,...,4}$ & normal & 0.22, 0.066 & [-6, 6] & 0.22\\
$\Delta_{\gcomm}$ & uniform & 0, 0.3 & [-6, 6] & 0\\
\hline 
\end{tabular} 
\vsd
\end{center}
\end{table}

\medskip

% \subsection{Description of deterministic components}
%\section{DESCRIPTION OF DETERMINISTIC COMPONENTS}
{\bf DESCRIPTION OF DETERMINISTIC COMPONENTS}

Notation (Table \AppLet.1) and set up of the deterministic components (Table \AppLet.2) are now described.

%currentRes = importRes("C:/Users/haighr/Files/GFish/PSARC13/SGR/Data/Awatea/CST/SGRrun16/SGR-CST.16.03.res", Dev=TRUE, CPUE=TRUE, Survey=TRUE, CLc=TRUE, CLs=TRUE, CAs=TRUE, CAc=TRUE, extra=TRUE)

%\section{Age classes}
\subsub{Age classes}

Index (subscript) $a$ represents age classes, going from 1 to the accumulator age class, $A$, of 32. Age class $a=5$, for example, represents fish aged 4-5 years (which is the usual, though not universal, convention, \citealt{casw01}), and so an age-class 1 fish was born the previous year. The variable $N_{ats}$ is the number of age-class $a$ fish of sex $s$ at the {\it start} of year $t$, so the model is run to year $T$ which corresponds to \finalYr.

%In \citet{\ymr, \popQCS} an accumulator age class of 60 was used, but this did not perform well in preliminary model runs for area 3CD, and so, in consultation with the Technical Working Group, the accumulator age class was set to 30. 

% \noindent {\bf Years}
% \headc
\subsub{Years}

Index $t$ represents model years, going from $1$ to $T=75$, and $t=0$ represents unfished equilibrium conditions. The actual year corresponding to $t=1$ is 1940, and so model year $T=75$ corresponds to \finalYr. 
%The model was run to the start of 2014 to incorporate the 2012 index from the West Coast Vancouver Island synoptic survey. 
Catch data for the whole of 2013 are not available (since the assessment model is being run in September 2013), and so the catch for 2013 was set to that for 2012. 

\subsub{Survey data}

Data from six survey series were used, as described in detail in Appendix~C. Here, subscript $g=1$ corresponds to the West Coast Haida Gwaii synoptic survey series, $g=2$ to the Hecate Strait synoptic survey, $g=3$ to the Queen Charlotte Sound synoptic survey, $g=4$ to the West Coast Vancouver Island synoptic survey, $g=5$ to the GB Reed historical survey series, and $g=6$ to the United States National Marine Fisheries Service Triennial survery series. The years for which data are available for each survey are given in Table \AppLet.1; ${\bf T}_g$ corresponds to years for the survey biomass estimates $I_{tg}$ (and corresponding standard deviations $\kappa_{tg}$), and ${\bf U}_g$ corresponds to years for proportion-at-age data $p_{atgs}$ (with assumed sample sizes $n_{tg}$). Note that there are no ${\bf U_5}$ or ${\bf U_6}$ because there are no age data for those surveys.

% Goose Island Gully historical survey, $g=2$ is the Queen Charlotte Sound synoptic survey, $g=3$ is the Queen Charlotte Sound shrimp survey, $g=4$ is the West Coast Haida Gwaii synoptic survey (note that Haida Gwaii was formerly known as the Queen Charlotte Islands) and $g=5$ is the West Coast Vancouver Island synoptic survey. 

\subsub{Commercial data}

As described in Appendix~B, the commercial catch has been reconstructed back to 1918. Given the negligible catches in the early years, the model was started in 1940, and catches prior to 1940 were not considered. The time series for catches is denoted $C_t$. The set ${\bf U}_\gcomm$ (Table \AppLet.1) gives the years of available ageing data from the commercial fishery. The proportions-at-age values are given by $p_{atgs}$ with assumed sample size $n_{tg}$, where $g=\gcomm$ (to correspond to the commercial data). These proportions are the weighted proportions calculated using the stratified weighting scheme described in Appendix~D, that adjusts for unequal sampling effort across temporal and spatial strata.

% All years are represented from 1978 to 2009, except for 1985, 1986 and 1988, and the
\subsub{Sex}

A two-sex model was used, with subscript $s=1$ for females and $s=2$ for males. Ageing data were partitioned by sex, as were the weights-at-age inputs. Selectivities and natural mortality were estimated by sex.

\subsub{Weights-at-age}

The weights-at-age $w_{as}$ are assumed fixed over time and based on the biological data. %; see Appendix D for details.

\subsub{Maturity of females}

The proportion of age-class $a$ females that are mature is $m_a$, and is assumed fix over time; see Appendix~D for details.

\subsub{State dynamics}

The crux of the model is the set of dynamical equations \eref{df1}-\eref{df3} for the estimated number $N_{ats}$ of age-class $a$ fish of sex $s$ at the start of year $t$. Equation \eref{df1} states that half of new recruits are males and half are females. Equation \eref{df2} calculates the numbers of fish in each age class (and of each sex) that survive to the following year, where $u_{ats}$ represents the proportion caught by the commercial fishery, and $e^{-M_s}$ accounts for natural mortality. Equation \eref{df3} is for the accumulator age class $A$, whereby survivors from this class remain in this class the following year.

Natural mortality $M_s$ was determined separately for males and females. It enters the equations in the form $e^{-M_s}$ as the proportion of unfished individuals that survive the year.
%  (note that e$^{-M_s} \approx 1 - M_s$ since $M_s$ is small).

\subsub{Initial conditions}

An unfished equilibrium situation at the beginning of the reconstruction is assumed, because there is no evidence of significant removals prior to 1940, and 1940 predates significant removals by about 15 years (Appendix~B). The initial conditions \eref{de1} and \eref{de2} are obtained by setting $R_t = R_0$ (virgin recruitment), $N_{ats} = N_{a1s}$ (equilibrium condition) and $u_{ats} = 0$ (no fishing) into \eref{df1}-\eref{df3}. The virgin spawning biomass $B_0$ is then obtained from \eref{dSt}. 

\subsub{Selectivities}

Separate selectivities were modelled for the commercial catch data and for each survey series. A half-Gaussian formulation was used, as given in \eref{self} and \eref{selm}, to give selectivities $s_{ags}$ (note that the subscript $\cdot_s$ always represents the index for sex, whereas $s_{...}$ always represents selectivity). This permits an increase in selectivity up to the age of full selection ($\mu_g$ for females). Given there was no evidence to suggest a dome-shaped function, it was assumed that fish older than $\mu_g$ remain fully selected. The rate of ascent of the left limb is controlled by the parameter $v_{gL}$ for females. For males, the same function is used except that the age of full selection is shifted by an amount $\Delta_g$, see \eref{selm}. 


\subsub{Derived states}

The spawning biomass (biomass of mature females, in tonnes) $B_t$ at the start of year $t$ is calculated in \eref{dSt} by multiplying the numbers of females $N_{at1}$ by the proportion that are mature ($m_a$), and converting to biomass by multiplying by the weights-at-age $w_{a1}$.

Equation \eref{uats} calculates, for year $t$, the proportion $u_{ats}$ of age-class $a$ and sex $s$ fish that are caught. This requires the commercial selectivities $s_{a\gcomm s}$ and the ratio $u_t$, which equation \eref{ut} shows is the ratio of total catch to vulnerable biomass in the middle of the year, $V_t$, given by equation (\ref{Vt}). So \eref{ut} calculates the proportion of the vulnerable biomass that is caught, and \eref{uats} partitions this out by sex and age.

\subsub{Stock-recruitment function}

A Beverton-Holt recruitment function is used, parameterised in terms of steepness, $h$, which is the proportion of the long-term unfished recruitment obtained when the stock abundance is reduced to 20\% of the virgin level \citep{md88, mm04}. This was done so that a prior for $h$ could be taken from \citet{fmdms10}. The formulation shown in (\ref{Rt}) comes from substituting $\alpha = (1 - h) B_0 / (4 h R_0)$ and $\beta = (5 h - 1) / 4 h R_0$ into the Beverton-Holt equation $R_t = B_{t-1} / (\alpha + \beta B_{t-1})$, where $\alpha$ and $\beta$ are from the standard formulation given in the Coleraine manual (\citealt{hmpeps03}; see also \citealt{mm04}), $R_0$ is the virgin recruitment, $R_t$ is the recruitment in year $t$, $B_t$ is the spawning biomass at the start of year $t$ and $B_0$ is the virgin spawning biomass. 
% Note that $R_1 = R_0$, and also $R_2 = R_0$ because $B_1 = B_0$.  - only for deterministic

% Have checked that B_{t-1} = B_0 / 5 into R_t equation gives h R_0, as per the definition.


\subsub{Estimates of observed data}

The model estimates of the survey biomass indices $I_{tg}$ are denoted $\widehat{I}_{tg}$ and are calculated in \eref{dg1}. The estimated numbers $N_{ats}$ are multiplied by the natural mortality term $e^{-M_s / 2}$ (that accounts for half of the annual natural mortality), the term $1 - u_{ats} / 2$ (that accounts for half of the commercial catch),  weights-at-age $w_{as}$ (to convert to biomass) and selectivity $s_{ags}$. The sum (over ages and sexes) is then multiplied by the catchability parameter $q_g$ to give the model biomass estimate $\widehat{I}_{tg}$. A 0.001 coefficient in \eref{dg1} is not needed to convert kg into tonnes, because $N_{ats}$ is in 1000s of fish (true also for \eref{dS0} and \eref{dSt}).

The estimated proportions-at-age $\widehat{p}_{atgs}$ are calculated in \eref{dg3}. For a particular year and gear type, the product $e^{-M_{s}/2} (1 - u_{ats}/2)  s_{ags} N_{ats}$ gives the relative expected numbers of fish caught for each combination of age and sex. Division by $\sum_{s=1}^2 \sum_{a=1}^A e^{-M_{s}/2} (1 - u_{ats}/2)  s_{ags} N_{ats}$ converts these to estimated proportions for each age-sex combination, such that $\sum_{s=1}^2 \sum_{a=1}^{A} \widehat{p}_{atgs} = 1$.

{\bf DESCRIPTION OF STOCHASTIC COMPONENTS}

\subsub{Parameters}

The set $\bfTh$ gives the parameters that are estimated. The estimation procedure is described in the Bayesian Computations section below.

%  ***  the stochastic version of the model. Estimation is accomplished by calculating the **residuals and maximising the log-likelihood function $\log L(\bfTh)$; in practice the negative log-likelihood $-\log L(\bfTh)$ is minimised in ADMB.

\subsub{Recruitment deviations}

For recruitment, a log-normal process error is assumed, such that the stochastic version of the deterministic stock-recruitment function (\ref{Rt}) is
\eb
R_t = \frac{B_{t-1}}{\alpha + \beta B_{t-1}} e^{\epsilon_t - \sigma_R^2/2}
\label{Rteqn}
\ee
where $\epsilon_t \sim \mbox{Normal}(0, \sigma_R^2)$, and the bias-correction term $- \sigma_R^2/2$ term in \eref{Rteqn} ensures that the mean of the recruitment deviations equals 0. This then gives the recruitment deviation equation (\ref{epst}) and log-likelihood function (\ref{ll1}). The value of $\sigma_R$ was fixed at 0.6, which is typical for marine {\it redfish} \citep{Metrz-Myers:1996}.
%was the value used in the Queen Charlotte Sound (QCS) POP assessment \citep{\popQCS}, where the value was determined empirically from model fits. 

% Tried 0.6, but that gave 0.8. 
% **PJS said: If $\sigma_R$ is estimated then its value can become large because it appears in the likelihood function [AME: but in (\ref{ll1}) it is always negative, so wouldn't that suggest that lower values would be preferred, to maximise the likelihood?]\marginpar{PJS, AME discuss}

\subsub{Log-likelihood functions}

The log-likelihood function \eref{ll2} arises from comparing the estimated proportions-at-age with the data. It is the Coleraine \citep{hmpeps03} modification of the \citet{fsmh90, fhs98} robust likelihood equation. The Coleraine formulation replaces the expected proportions $\widehat{p}_{atgs}$ from the \citet{fsmh90, fhs98} formulation with the observed proportions $p_{atgs}$, except in the $( p_{atgs} - \widehat{p}_{atgs} )^2$ term \citep{bfdmgs05}.

% Equation \eref{ll2} is slightly different to that originally given by Hilborn {\it et al.}~(2003)\nocite{hmpeps03}, but agrees with the Coleraine code, and with the discussions of various likelihood formulations by Bull {\it et al.}~(2005)\nocite{bfdmgs05}, and the description by Stanley {\it et al.}~(2009)\nocite{sso09}. 

The $1/(10 A)$ term in \eref{ll2} reduces the weight of proportions that are close to or equal zero. The $1/100$ term reduces the weight of large residuals $( p_{atgs} - \widehat{p}_{atgs} )$. The net effect (Stanley {\it et al.} 2009)\nocite{sso09} is that residuals larger than three standard deviations from the fitted proportion are treated roughly as $3 (p_{atgs} (1 - p_{atgs}))^{1/2}$. 

% Don't think you can easily write down a residual equation for the $p_{atgs}$ terms. Just write down the Fournier {\it et al.}~(1990) log-likelihood function. That is equation (2.3) in their paper, here replacing $Q_{i\alpha}$ with $p_{atgs}$ and other changes....

Lognormal error is assumed for the survey indices, resulting in the log-likelihood equation \eref{ll3}. The total log-likelihood $\log L(\bfTh)$ is then the sum of the likelihood components -- see \eref{llk}.

% There is no equivalent to the bias-correction term $- \sigma_R^2/2$ for recruitment, because the standard deviations $\kappa_{tg}$ are different for each data value.



% \subsection{Further details of Bayesian aspects\label{sec:Bayes}}

% \clearpage     % To bump to next page
{\bf BAYESIAN COMPUTATIONS}

Estimation of parameters compares the estimated (model-based) observations of survey biomass indices and proportions-at-age with the data, and minimises the recruitment deviations. This is done by minimising the objective function $f(\bfTh)$, which equation \eref{objfn} shows is the negative of the sum of the total log-likelihood function and the logarithm of the joint prior distribution, given by \eref{jointprior}.

The procedure for the Bayesian computations is as follows:
\begin{enumerate}
\item minimise the objective function $f(\bfTh)$ to give estimates of the mode of the posterior density (MPD) for each parameter

\begin{itemize}

\item this is done in phases  % {\hskip 10mm}   -- 
   
\item a reweighting procedure is performed 

\end{itemize}

\item generate samples from the joint posterior distributions of the parameters using Monte Carlo Markov Chain (MCMC) procedure, starting the chains from the MPD estimates.
\end{enumerate}

%The details for these steps are now given.

\subsub{Phases}

The MPD estimates were obtained by minimising the objective function $f(\bfTh)$, from the stochastic (non-Bayesian version) of the model. The resulting estimates were then used to initiate the chains for the MCMC procedure for the full Bayesian model.

Simultaneously estimating all the estimable parameters straight away for complex nonlinear models is ill advised, and so ADMB allows some of the estimable parameters to be kept fixed during the initial part of the optimisation process \citet{ADMB2009}. Some parameters are estimated in phase 1, then some further ones in phase 2, and so on. The order used here was:

phase 1: virgin recruitment $R_0$ and survey catchabilities $q_{1,...,6}$~;

phase 2: recruitment deviations $\epsilon_t$ (held at 0 in phase 1);

phase 3: age of full selectivity for females $\mu_{1,...,4,\gcomm}$~;

phase 4: natural mortality $M_{1,2}$ and selectivity parameters $\Delta_g, v_{gL}$ for $g = 1,...,4,7$;

phase 5: steepness $h$.

\subsub{Reweighting}

Given that sample sizes are not comparable between different types of data, a procedure that adjusts the relative weights between data sources is required. The QCS POP assessment \citep{\popQCS} used an iterative reweighting scheme based on adjusting the standard deviation of normal residuals (SDNRs) of data sets until these standard deviations were approximately 1. This procedure did not perform well for the Yellowmouth Rockfish assessment \citep{\ymr}, leading to spurious cohorts; therefore, the Yellowmouth assessment used the reweighting scheme proposed by \citet{fran11}. In this assessment, we use the latter scheme -- weighting age sample size by mean age (see below).

For abundance data such as survey indices, \citet{fran11} recommends reweighting observed coefficients of variation, $c_0$, by first adding process error $c_\mathrm{p} = 0.2$ to give a reweighted coefficient of variation
\eb
c_1 = \sqrt{c_0^2 + c_\mathrm{p}^2}~.
\label{reweight}
\ee
For each survey index, $I_{tg}$ ($g=1,...,4; t \in {\bf T}_g$), the associated standard deviation is $\kappa_{tg}$. The associated coefficient of variation is therefore $\kappa_{tg} / I_{tg}$, which is used in \eref{reweight} to determine the reweighted coefficient of variation associated with $\kappa_{tg}$. This reweighted coefficient of variation is then converted back to a standard deviation, which is used as the reweighted standard deviation $\kappa_{tg}$ in the likelihood function \eref{ll3}. In this assessment, we manually adjusted the CV process error at each reweight until the surveys attained SDNR values between 0.8 and 1.2. This was achieved after three reweights. We placed an upper limit of 0.4 to the added process error to avoid completely devaluing the survey.

% is applied to for each combination $I_{tg}$. , this calculation is done for each standard deviation $\kappa_{tg}

\citet{fran11} maintains that correlation effects are usually strong in age-composition data. Each age-composition data set has a sample size $n_{tg}$ ($g=1,4$, $t \in {\bf U}_g$), which is typically in the range 3-20. Equation (T3.4) of \citet{fran11} is used to iteratively reweight the sample size as
\eb
n_{tg}^{(r)} = W_{g}^{(r)} n_{tg}^{(r-1)}
\ee
where $r=1,2,3,...,N$ represents the reweighting iteration, $n_{tg}^{(r)}$ is the effective sample size for reweighting $r$, $W_{g}^{(r)}$ is the weight applied to obtain reweighting $r$, and $n_{tg}^{(0)} = n_{tg}$. So a single weight $W_{g}^{(r)}$ is calculated for each series $g=1,...,4,\gcomm$ for reweighting $r$.

The \citet{fran11} weight $W_{g}^{(r)}$ given to each data set takes into account deviations from the mean weight for each year, rather than the scheme used for the QCS POP assessment \citep{\popQCS} that considered deviations from each proportion-at-age value. It is given by equation (TA1.8) of \citet{fran11}:
%\eb
%W_{g}^{(r)} = \left\{ \Var_t \left[ \frac{\bar{O}_{gt} - \bar{E}_{gt}}{\sqrt{ \theta_{gt} / n_{tg}^{(r-1)}}} \right] \right\}^{-1}
%\ee
\eb
W_{g}^{(r)} = \left\{ \Var_t \left[ \frac{\overline{O}_{gt} - \overline{E}_{gt}}{\sqrt{ \theta_{gt} / n_{tg}^{(r-1)}}} \right] \right\}^{-1}
\ee
where the observed mean age, the expected mean age and the variance of the expected age distribution are, respectively,
%\eb
%\bar{O}_{gt} & = & \sum_{a=1}^{A} \sum_{s=1}^2 a p_{atgs}\\
%\bar{E}_{gt} & = & \sum_{a=1}^{A} \sum_{s=1}^2 a \widehat{p}_{atgs}\\
%\theta_{gt} & = & \sum_{a=1}^{A} \sum_{s=1}^2 a^2 \widehat{p}_{atgs} - \bar{E}_{gt}^2
%\ee
\eb
\overline{O}_{gt} & = & \sum_{a=1}^{A} \sum_{s=1}^2 a p_{atgs}\\
\overline{E}_{gt} & = & \sum_{a=1}^{A} \sum_{s=1}^2 a \widehat{p}_{atgs}\\
\theta_{gt} & = & \sum_{a=1}^{A} \sum_{s=1}^2 a^2 \widehat{p}_{atgs} - \overline{E}_{gt}^2
\ee
and $\Var_t$ is the usual finite-sample variance function applied over the index $t$. For the Yellowmouth Rockfish assessment \citet{\ymr} we used this approach iteratively with $r=1,2,...,6$, but found that reweightings after the first ($r=1$) had only a marginal effect; the reported results for this assessment were based on the third reweighting.

% \subsub{OLD from POP, presumably replace most of this with Rowan's:}

% POP: The standard deviation of normal residuals (SDNR), or standard deviation of Pearson residuals, was calculated for each data source (three series of biomass estimates from surveys and the commercial and survey proportions-at-age data). Successive fits of a given model involved adjusting the relative weights until SDNR values close to 1 were obtained for each data source (the 1 comes from a standard normal distribution having a mean zero and a standard deviation of 1).

% POP: In general, the normal residual for an observation $i$ is
%\eb
%r_i = \frac{O_i - P_i}{d(O_i)},
%\ee
%where $O_i$ is the observed value, $P_i$ is the predicted value, and $d(O_i)$ is the standard deviation associated with the observed value. 

% For the normal and lognormal error distributions, the standard deviation is calculated as
%\eb
%d(O_i) = P_i C_i
%\ee
% where $C_i$ is the coefficient of variation of the observation (Bull $et~al.$, 2005, section 6.8)\nocite{bfdmgs05}. 

%POP: Each survey biomass estimate $I_{tg}$ has an associated standard deviation $\kappa_{tg}$, so the resulting normal residual $r_{tg}$ is
%\eb
%r_{tg} = \frac{I_{tg} - \widehat{I}_{tg}}{\kappa_{tg}}.
%\ee
%For each survey series $g=1,2,3$, the standard deviation of the normal residuals $r_{tg}$ was then calculated. This thus results in three SDNR values.

% For each survey index series $\{ I_{tg} \}_{t \in {\bf T}_g}$, the standard deviation is calculated across the years, and lognormal error is assumed. So the residual $r_{tg}$ for observation $I_{tg}$ is
%\eb
%r_{tg} = \frac{I_{tg} - \widehat{I}_{tg}}{d_g}
%\ee
%where $d_g$ is the standard deviation calculated for series $g$.

% an observation $i$ is a vector of proportions with sample size $N_i$ and number of age bins $A$. 

%POP: For the proportions-at-age data, the robust normal likelihood function is used. For a given year $t$ and data series $g$, the standard deviation (used to calculate the normal residuals) of the observed proportion-at-age, $p_{atgs}$, can be written 
%\eb
%d(p_{atgs}) = \sqrt{ \left. \left(  p_{atgs}(1 - p_{atgs}) + \frac{1}{10 A} \right) \right/ N'_{tg} }
%\ee
%where $N'_{tg} = \min (N_{tg}, 200)$, for which $N_{tg}$ is the effective sample size. Initially, $N_{tg} = 1/\tau_{tg} $. The definition of $N'_{tg}$ gives a maximum effective sample size of 200, which avoids putting excessive weight on any one set of age composition data. Also, any standardised residual $>3$ was set to 3.

% POP: For the ageing data from the commercial catch, a single SDNR was calculated as the standard deviation of all the normal residuals for all the data. For the ageing data from surveys, a single SDNR was calculated as the standard deviation of all the normal residuals for all the survey ageing data combined. Thus, there were five SDNR values in total (one for each survey index series, one for the commercial age data and one for the survey age data).

% POP: The standard deviations $\kappa_{tg}$ and effective sample sizes $N_{tg}$ are essentially weights associated with each data set. The weights were iteratively adjusted manually until each SDNR was was approximately 1.0, consistent with the error assumptions. If the SDNR for a data set was $<$1.0, the data set was judged to have too little weight (hence the weight for that data set was increased), while the opposite was true if SDNR was $>$1.0. The SDNRs for the age composition data could not reach 1.0 because of the truncation to 3, and the maximum effective sample size of 200. Each model run described in Appendix G was independently reweighted, yet all model runs resulted in similar weighting terms and SDNRs. Note that the recruitment deviations are not part of the reweighting process, because there are no associated sample sizes from data. 

% Handwritten notes on earlier say:  SDNR for each:

%  survey index - sd over years separately for each g

% age data surveys - sd over years for all surveys together

% age data commercial - sd over years

% ...The SDNR will not reach 1 ... -- for the proportion-at-age commercial data?   Because SDNR's $>$3 are set to 3, and also sample sizes are truncated to 200.

% `` in particular point out that recruitment deviations are not part of reweighting, and may want to justify a bit more why it's a reasonable thing to do (alternative is not doing it, but then the sample sizes are not comparable between different types of data ''

% Rough notes:


% Empirical sample size calculation - uses how well it fits the data, forgets about the initial sample size (which Rick thought problematic when newer data got downweighted because of the anomalous 1984 year class - that's all cleared when we took that data set out).    

% If  $tau_{tg}$ big then the effective weight on the likelihood is small for that $t,g$. The reweighting drives survey deviations to N(0,1). 

% What happens with the age residuals happens - the recruitment deviations are not part of the reweighting (I think, check documentation). Ianelli and McAllister.  

\subsub{Prior distributions}

Descriptions of the prior distributions for the 26 estimated parameters (without including recruitment deviations) are given in Table \AppLet.4 and in Table 1A and Table 1B in the Main Document. The resulting probability density functions give the $\pi_j(\bfTh)$, whose logarithms are then summed in \eref{jointprior} to give the joint prior distribution $\pi(\bfTh)$. Since uniform priors are, by definition, constant across their bounded range (and zero outside), their contributions to the objective function can be ignored. Thus, in the calculation \eref{jointprior} of the joint prior distribution $\pi(\bfTh)$, only those priors that are not uniform need to be considered in the summation.


% \subsubsection*{Age classes}   % then doesn't indent text
% \noindent {\bf Age classes}
% \headc

%A uniform prior over a large range was used for $R_0$. The priors for female and male natural mortality, $M_1$ and $M_2$ respectively, were based on the results of the QCS POP assessment \citep{\popQCS}. We first fit normal distributions, using maximum likelihood, to the posteriors from the `Estimate $M$ and $h$' model run of \citet{\popQCS}, yielding N(0.0668, 0.00293) [indicating a normal distribution with mean 0.0668 and standard deviation 0.00293] for females, and N(0.0727, 0.00314) for males. For the QCS POP assessment we had taken priors from the posterior distributions of the Gulf of Alaska assessment of POP \citep{hans07, hans09}, namely N(0.06, 0.006) [rounding the mean to one decimal place] for both females and males. To avoid the overly tight priors based on our likelihood analysis, we set the coefficient of variation here to 0.1 (the same as the Gulf of Alaska value). Given the closeness of the resulting female and male distributions, with an overall mean of 0.0697, for the current assessment we used a single prior for females and males of N(0.07, 0.007).

%For steepness, $h$, the same prior was used as for the QCS POP assessment \citep{\popQCS} -- a beta distribution with values fitted to the posterior distribution for rockfish calculated by \citet{fmdms10}, with the Pacific ocean perch data removed (R.~Forrest, DFO, pers.~comm., though removing those data made little difference to the distribution). Uniform priors on a logarithmic scale were used for the catchability parameters $q_g$. 

%Selectivity was estimated for the West Coast Vancouver Island synoptic survey series ($g=1$), because age data were available. Priors for the three selectivity parameters, $\mu_1, \Delta_1$, and $v_{1L}$ were based on the results from the QCS POP assessment \citep{\popQCS}. Normal distributions were used for the priors, with means taken from the median values of the posteriors for the QCS synoptic survey series for the `Estimate $M$ and $h$' model run, given as $\mu_2=13.3, \log v_{2L}=3.30$ and $\Delta_2 =0.22$ in Table G3 (p156) of \citet{\popQCS}. To give broad priors here, the standard deviations of the priors were set to give coefficients of variation of 0.3. 

%For the other two survey series, the National Marine Fisheries Service triennial survey series and the GB Reed historical survey series, no age data were available, and so the selectivity parameters were held fixed rather than estimated. The aforementioned median values were used for the fixed values.

%For the commercial selectivity ($g=4$), age data were available and so selectivity was estimated. Again, the priors for the three parameters were normal distributions with means based on the median values of the posteriors for the `Estimate $M$ and $h$' model run of the QCS assessment, given in Table G3 (p156) of \citet{\popQCS} as $\mu_4=10.5, \log v_{4L}=1.52$ and $\Delta_4 =0.00$ . To give broad priors, the standard deviations of the priors were set to give coefficients of variation of 0.3 (except for $\Delta_4$ for which the standard deviation was set to 0.3, because its mean was 0).

A uniform prior over a large range was used for $R_0$. 
The priors for female and male natural mortality, $M_1$ and $M_2$ respectively, were based on previous assessments of Silvergray that assume $M$~= 0.06 \citep{Stanley-Kronlund:2000, Stanley-Olsen:2002}, which we use as the mean and assume a 10\% CV (Table~\AppLet.4).

%***** PAUL something here. Last time AME said:\\
For steepness, $h$, the same prior was used as for the QCS POP assessment \citep{\popQCS} -- a beta distribution with values fitted to the posterior distribution for rockfish calculated by \citet{fmdms10}. Uniform priors on a logarithmic scale were used for the catchability parameters $q_g$. 

%***** PAUL something here. Last time AME said:\\
Selectivity was estimated for the  four surveys with age composition data: west coast Haida Gwaii, Hecate Strait, Queen Charlotte Sound and west coast Vancouver Island synoptic survey series ($g=1$ to $g=4$). Informative priors were developed for the three selectivity parameters for each of these surveys, $\mu_{1,...,4}, \Delta_{1,...,4}$, and $v_{1,...,4L}$, based on the median values for the same parameters from the matching base case POP assessments \citep{\popQCS, \popWCVI, \popWCHG}. The parameter estimates from the west coast Haida Gwaii synoptic survey were used for the Hecate Strait survey because this survey was not used in these POP assessments. Normal distributions were assumed for the priors, with the means taken from the median values of the posterior distributions and with the standard deviations set to give coefficients of variation of 0.3. 

No age data were available for the other two survey series, the GB Reed historical survey series and the National Marine Fisheries Service triennial survey series, so the three selectivity parameters for these surveys were fixed rather than estimated. The fixed values used for these selectivities were the posterior medians from the same survey in the 5ABC base case POP stock assessment (for the GB Reed survey series) and from the WCVI synoptic survey from the 3CD base case POP stock assessment (for the Triennial survey series).

%***** PAUL something here. Last time AME said:\\
For the commercial selectivity ($g=\gcomm$) the priors for the three parameters were uniform (non-informative) distributions with starting values based on the median values of the posterior disributions for the `Estimate $M$ and $h$' model run of the 5ABC POP stock assessment. 
%means based on the median values of the posterior disributions for the `Estimate $M$ and $h$' model run of the 5ABC POP stock assessment and with the standard deviations set to give coefficients of variation of 0.3.

% Priors taken from InputPOPassess7.xls, which was used for four runs, though h and M_s priors given in Paul's talk  

% \subsub{Model runs}

% Results from four model runs are presented in this assessment, resulting from a combination of fixing or estimating steepness $h$ and mortalities $M_s$, as described in the main text.

% . Details are given in Table \AppLet5.

% \begin{table}[tp]
% {\it Table \AppLet5 Descriptions of four model runs.}
%\vsd \\
%\begin{tabular}{ll} 
% \multicolumn{2}{l} {Table \AppLet5 Descriptions of four model runs.}\\
% & \\
%\hline
%Run & Description \\ 
%\hline %\\[-.5ex]
%
%Estimate $M_s$ and $h$ & normal prior for $M_s$, beta prior for $h$\\
%Fix $M_s$ and $h$ & $M_s$ fixed at 0.06, $h$ fixed at 0.0674\\
%Estimate $M_s$ & normal prior for $M_s$, $h$ fixed at 0.0674\\
%Estimate $h$ & $M_s$ fixed at 0.06, beta prior for $h$\\
%\hline
%\end{tabular}	
%\end{table}

% The female and male mortalities $M_1$ and $M_2$ are fixed at the same value of 0.06 (the mean of the priors just discussed) when they not estimated. Steepness $h$ was fixed at the mean of the Forrest {\it et al.}~(2010) prior when it was not estimated.
% \clearpage
\vsd
\vsd   % to force subsub onto next page. Should have redefined correctly as a subheading.

\subsub{MCMC properties}

The MCMC procedure started the search from the MPD values and performed 10,000,000 iterations, sampling every 10,000$^\mathrm{th}$ for 1,000 samples, which were used with no burn-in period (because the MCMC searches started from the MPD values).  
% \clearpage % to get heading on new page

{\bf REFERENCE POINTS, PROJECTIONS AND ADVICE TO MANAGERS}

Advice to managers is given with respect to two sets of reference points or reference criteria. The first set consists of the provisional reference points of the DFO Precautionary Approach \citep{dfo06}, namely 0.4\Bmsy~ and 0.8\Bmsy~ (and we also provide \Bmsy); \Bmsy{} is the estimated equilibrium spawning biomass at the maximum sustainable yield (MSY). The second set of reference points comprises 0.2$B_0$ and 0.4$B_0$, where $B_0$ is the estimated unfished equilibrium spawning biomass. See main text for further discussion.
% The reference criteria are defined in terms of a changing reference biomass, $B_{t-3Gen}$ (the spawning biomass three generations before $B_t$, which is itself the spawning biomass at the beginning of year $t$).

To estimate \Bmsy, the model was projected forward across a range (0 to 0.3 incremented by 0.001) of constant harvest rates ($u_t$), for a maximum of 15,000 years until equilibrium was reached (with a tolerance of 0.01~t). The MSY is the largest of the equilibrium yields, and the associated exploitation rate is then \umsy~ and the associated spawning biomass is \Bmsy. This calculation was done for each of the 1,000 MCMC samples, resulting in marginal posterior distributions for MSY, \umsy~ and \Bmsy. 

% For run \EstM, an equilibrium was not reached for only 3 out of the 1,000 MCMC samples (the maximum value of $u_t=0.3$ was attained). For the other 997 samples (and for all 1,000 samples for run \FixM), equilibrium was reached (the model was run for a maximum of 15,000 years with a 0.01 tolerance for defining that equilibrium yield had been reached). 

The probability P$(B_{\finalYr} > 0.4 B_\mathrm{MSY})$ is then calculated as the proportion of the 1,000 MCMC samples for which $B_{\finalYr} > 0.4 B_\mathrm{MSY}$ (and similarly for the other reference points).

Projections were made for 10 years %(as requested by N.~Davis, DFO Groundfish Management Unit), 
starting with the biomass and age structure calculated for the start of \finalYr.  A range of constant catch strategies were used, from 0-3,000~t (the average catch from 2009-2013 was 1431~t). For each strategy, projections were performed for each of the 1,000 MCMC samples (resulting in posterior distributions of future spawning biomass). Recruitments were randomly calculated using \eref{Rteqn} (i.e.~based on lognormal recruitment deviations from the estimated stock-recruitment curve), using randomly generated values of $\epsilon_t \sim \mbox{Normal}(0, \sigma_R^2)$. For each of the 1,000 MCMC samples a time series of $\left\{ \epsilon_t \right\}$ was generated. For each MCMC sample, the same time series of $\left\{ \epsilon_t \right\}$ was used for each catch strategy (so that, for a given MCMC sample, all catch strategies experience the same recruitment stochasticity).

% Note that these projections, given the longevity and consequent low natural mortality rate, will make use of year classes which were estimated during the stock reconstruction and that none of the new randomly generated recruitments will affect the early projection years (because the randomly generated recruits will be too young to become part of the mature or vulnerable biomasses by the end of the five-year projection period).  

% Next paragraph was written for YMR11, but not actually implemented every year - for POP, Paul just calculated the values for 2011. Would have to code up separately from Awatea I think (could use determinR code used to check Steve's issue.
% An additional factor is that yield each year must satisfy a harvest control rule to be consistent with the Precautionary Approach (DFO 2006)\nocite{dfo06}. The equation used to calculate the Precautionary-Approach compliant maximum yield, $Y_t$, for year $t$ is
% \eb
% Y_t = \left\{ 
% \begin{array}{ll}
% 0, & B_t \leq 0.4 B_{MSY}\\
% \frac{(B_t - 0.4 B_{MSY})}{0.4 B_{MSY}}  u_{MSY} V_t, & 0.4 B_{MSY} < B_t \leq 0.8 B_{MSY}\\
% u_{MSY} V_t, & 0.8 B_{MSY} < B_t
%\end{array}
%\right.
%\ee
%where $V_{MSY}$ is the vulnerable biomass at the maximum sustainable yield, and we have assumed a harvest policy that declines linearly between the reference points $0.4 B_{MSY}$ and $0.8 B_{MSY}$. This is the provisional harvest rule described in DFO (2009). Thus, if the stock is in the critical zone ($B_t \leq 0.4 B_{MSY})$ then no catch is allowed, if it is in the healthy zone $(B_t > 0.8 B_{MSY})$ then a maximum exploitation rate of u${MSY}$ is allowed, and if it is in the in-between cautious zone $(0.4 B_{MSY} < B_t \leq 0.8 B_{MSY})$ there is a proportionate linear reduction in allowable yield. For projections (which use a constant catch strategy), the actual yield is then the minimum of $Y_t$ and the constant catch strategy.

% Therefore, if a significant amount of the weight of the posterior distribution of $B_t$ lies above $B_{MSY}$ then the median value for $P_t$ can be greater than the median value of the maximum sustainable yield.
 
% APP F: For each constant catch strategy, projections were made for each of the 1000 MCMC samples, resulting in a posterior distribution of 1000 estimates of biomass for each year.  The catch in each year was reduced according to the Precautionary Approach....... MERGE with text with equation F.28 in Appendix F).

% The PA-compliant yields are calculated based on UMSY  (the exploitation rate associated with the MSY).  If the stock is above the upper stock reference level, then UMSY is applied to the vulnerable biomass to calculate the potential yield.  If the stock is in the "cautious zone", then UMSY is discounted proportionately, relative to how far the stock is below the upper reference level, until the limit reference point is reached, before multiplying by the stock size to estimate the yield.  If the stock size is below the limit reference point, yield is set equal to 0.

\newpage
%\bibliography{../../../abbrev3}
%\bibliography{../SGR_2013}
\bibliographystyle{C:/Users/haighr/Files/Archive/Latex/ecology}
\bibliography{C:/Users/haighr/Files/GFish/PSARC13/SGR/Docs/SGR_2013}


\end{document}

\subsub{Awatea notation}

For reference when using Awatea code, Table \AppLet.5 gives the names of estimated parameters used here with the respective names from Awatea input sheet and the Awatea output that is imported into R using programs modified from the {\tt scape} package written by A. Magnusson (University of Washington). 
% 2009, )\nocite{magnu09}.

\begin{table}[tp]
{\it Table \AppLet.5 Names of the 16 estimated parameters with corresponding Awatea names for reference. In Awatea, some parameters are always written as the log of that parameter, (though note that for $v_{gL}$, $g=1,2$, the input name does not explicitly mention that it is the logarithm, but it does for $g=4$).} \vsd
\\
\begin{tabular}{lll} 
\hline
Parameter & Awatea input name & Awatea export name\\ 
\hline %\\[-.5ex]
%
$\mu_1$ & Survey L full, row 1 & surveySfull\_1\\
$\mu_2$ & Survey L full, row 2 & surveySfull\_2\\
% $\mu_3$ & Survey L full, row 3 & surveySfull\_3\\
$\mu_4$ & S fullest & Sfullest\_1 \\
$\log v_{1L}$ & Survey variance L, row 1 & log\_surveyvarL\_1\\
$\log v_{2L}$ & Survey variance L, row 2 & log\_surveyvarL\_2\\
% $\log v_{3L}$ & Survey variance L, row 3 & log\_surveyvarL\_3\\
$\log v_{4L}$ & Log variance of left side of selectivity & log\_varLest\_1\\
              & ~~curve by length (for both sexes) & \\
$\Delta_1$ & Survey L full delta & surveySfulldeltaest\_1\\
$\Delta_2$ & Survey L full delta & surveySfulldeltaest\_2\\
% $\Delta_3$ & Survey L full delta & surveySfulldeltaest\_3\\
$\Delta_4$ & S fullest & Sfulldelta\_1\\
$\log q_1$ & Log q survey, row 1 & log\_qsurvey\_1\\
$\log q_2$ & Log q survey, row 2 & log\_qsurvey\_2\\
$\log q_3$ & Log q survey, row 3 & log\_qsurvey\_3\\
$M_1$ & M (natural mortality), row 1 & M1\_1 \\
$M_2$ & M (natural mortality), row 2 & M1\_2 \\
$R_0$ & R0 (recruitment in virgin condition) & R0\\
$h$ & h (steepness of spawner-recruit curve) & h\\
\hline
\end{tabular}	
\end{table}


% \subsection{To check with Paul/Rowan}

% PJS - should $A$ in the two denominators in \eref{ll2} actually be $2A$? I'm thinking this would be because the sample size is $2A$ due to it being two-sex model, and the proportions are such that $\Sigma_{a=1}^{A} \Sigma_{s=1}^2 p_{atgs} = 1$. If time we could check the code.
% We looked at code, it is 10A and 100A.

% PJS - Do you have the justifications for the priors already written up? 

% Uncertainty in the initial age structure is also incorporated using log-normal error. Thus the residual equations (\ref{xias}) and (\ref{xiAs}) are derived similarly to (\ref{epst}) above, to give the log-likelihood function (\ref{ll2}). **The standard deviation $\sigma_I$ is assumed the same for all age classes and both sexes, as the Awatea input form only has ``deviates for initial age structure'', and manual is confusing.




% \bibliographystyle{ecologyfulljournal}  % naturemag
% \bibpunct{(}{)}{,}{a}{}{,}   % see natbib.pdf, p13
% \bibliographystyle{natbibfull}
%\bibliographystyle{ecology}   % ecology for now
%\bibliography{../../../abbrev3}

% To do, then straight into Word.\marginpar{AME}.
% Include some of the following when write up fully.


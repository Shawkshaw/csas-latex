% AppD-BiologyRev.tex - revisions for Res Doc. Adding in figure of
%  weights-at-age. 27th May 2013.
% AppD-Biology.tex - splitting what was E back into D (biology) 
%  and E (ages). So this is just from Andy's AppE-extra.tex

% AppE-Biology.tex. From Andy's AppE-extra.tex and Rowan's 
%  AppE-Biology.tex from his Sweave. 2nd October 2012.

%\usepackage{graphicx}
%\usepackage{verbatim,fancyvrb}
%\usepackage{hyperref,url} %,parskip} %,inconsolata}

%\usepackage[utf8]{inputenc}
%\usepackage{mathtools}
%\usepackage[multiple]{footmisc}
%\usepackage{longtable,array} % need array when specifying a ragged right column:  >{\raggedright\arraybackslash}{p2in}
%\usepackage{xifthen}         % provides \ifthenelse and \isempty

\clearpage

\chapter{BIOLOGICAL DATA}      % This will get numbered automatically

Here we present the parameterisation of the weights-at-age and female maturity. These are taken from our Pacific Ocean Perch assessment for Queen Charlotte Sound \citep{\popQCS}; the same values are used for the companion assessment \citep{\otherCite} for area \other.

\section{Parameterisation of weights-at-age}

The estimates of weights-at-age are the same as those used in \citet{\popQCS}. The average weight of an individual of age-class $a$ of sex $s$ is denoted $w_{as}$ (kg), and is given by
\eb
w_{as} = \alpha_s L_{as}^{\beta_s},
\label{weight}
\ee
where (for each sex, $s$) $\alpha_s$ is the growth rate scalar, $L_{as}$ is the length (cm) of an individual of age $a$, and $\beta_s$ is the growth rate exponent. Sex $s=1,2$ for females and males, respectively.

The lengths $L_{as}$ are given by the von-Bertalanffy model
\eb
L_{as} = L_{\infty,s} \left( 1 - e^{-k_s (a - t_{0,s} ) } \right),
\label{vonB}
\ee 
where (for each sex, $s$) $L_{\infty,s}$ is the average length at maximum age of an individual, $k_s$ is the growth rate coefficient, and  $t_{0,s}$ is the age at which the average size is zero.

In \citet{\popQCS} data came from Queen Charlotte Sound and the west coasts of Vancouver Island and Haida Gwaii. The differences between areas were found to be relatively small and probably a result of data issues rather than reflecting actual differences in growth rates among the five areas. There was also little sensitivity to combining length-age pairs from research and commercial sources or using each source separately. The parameters were calculated from combining data from areas 5A, 5B, 5C and 5E. Given the similarities between areas, here we use the same estimated parameters as in \citet{\popQCS}. The values are given in Table \ref{tab:biolPars}, and were used as fixed inputs to the stock assessment model (to calculate $w_{as}$ for each $a$ and $s$). Figure \ref{fig:lengthweight} shows the resulting mean lengths-at-age and mean weights-at-age.



\begin{table}[b]                 % p for next page, tp for top of 
\centering                        %  current page if possible
\caption{\label{tab:biolPars} Fixed allometric growth parameters for females and males, used as inputs for the stock assessment model. See text for parameter definitions.}
\begin{tabular}{lrr} 
\hline
Parameter & Females & Males\\
\hline 
$L_{\infty,s}$ & 45.11 & 41.62 \\
$k_s$ & 0.1404 & 0.1675 \\
$t_{0,s}$ & $-$1.303 & $-$1.021 \\
$\alpha_s$ & 9.258$\times$10$^{-6}$ & 8.126$\times$10$^{-6}$\\
$\beta_s$ & 3.116 & 3.155 \\
\hline
\end{tabular}	
\end{table}

% Figure created in:
%  C:\andy\awatea\POP12\POP3CD\weight.Snw

\begin{figure}[htp]
\begin{center}
\epsfbox{appD-Biology/lengthweight.eps}
\vspace{-12mm}   % seems to need nudging up a bit more
\end{center}
\caption{Mean lengths-at-age and mean weights-at-age for each sex, as given by (\ref{vonB}) and (\ref{weight}) with parameter values from Table \ref{tab:biolPars}.}
\label{fig:lengthweight} 
\end{figure}


\section{Parameterisation of female maturity}

The proportion of age-class $a$ females that are mature, $m_a$, was also taken from \citet{\popQCS}. The resulting ogive is given in Figure \ref{fig:maturity} and Table \ref{tab:maturity}, and was based on 21,000 observations from PMFC areas 5ABCE. It was calculated by fitting a double-normal function to the observed proportions that were mature at each age, and then, as for \citet{sso09} for Canary Rockfish, using the observed proportions for ages $<9$ because the fitted function appeared to overestimate the proportion of mature females.

% \vspace{-10mm}
% Figure and table created in 
%  C:\awatea\POP12\POP3CD\POP3CDmaturity.Snw.
%  C:\andy\awatea\POP12\POP3CD\maturity.Snw - 2013 tecra.
\begin{figure}[tp]
\begin{center}
% \centering
\epsfxsize=3.5 in
\epsfbox{appD-Biology/maturity.eps}
\vspace{-8mm}
\end{center}
\caption{Maturity ogive for females used in the stock assessment model. Values are given in Table \ref{tab:maturity}.}
\label{fig:maturity} 
\end{figure}

\begin{table}[btp]
\begin{center}
\caption{Maturity ogive for females used in the stock assessment model, as plotted in Figure \ref{fig:maturity}.}
\label{tab:maturity}
\begin{tabular}{rrcrr}
  \hline
Age & Proportion & ~ & Age & Proportion\\ 
 & mature & & & mature \\
  \hline
1 & 0.000 & ~ & 11 & 0.601 \\ 
2 & 0.000 & ~ & 12 & 0.738 \\ 
3 & 0.000 & ~ & 13 & 0.860 \\ 
4 & 0.000 & ~ & 14 & 0.950 \\ 
5 & 0.023 & ~ & 15 & 0.996 \\ 
6 & 0.034 & ~ & 16 & 1.000 \\ 
7 & 0.096 & ~ & 17 & 1.000 \\ 
8 & 0.211 & ~ & 18 & 1.000 \\ 
9 & 0.341 & ~ & 19 & 1.000 \\ 
10 & 0.465 & ~ & 20 & 1.000 \\
\hline
\end{tabular}
\end{center}
\end{table}

\clearpage

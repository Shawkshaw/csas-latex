\clearpage

\chapter{WEIGHTING OF AGE PROPORTIONS}
\label{chap:agecompweight}

This appendix summarizes a method for representing commercial and survey age structures for a given species through weighting observed age frequencies~$x_a$ or proportions~$x^\prime_a$ by \eor{catch}{density} in defined strata. 
(Throughout this section, we use the symbol \sQuote{$\Vert$} to delimit parallel values for commercial and survey analyses, respectively, as the mechanics of the weighting procedure are similar for both.) 
For commercial samples, these strata comprise quarterly periods within a year, while for survey samples, the strata are defined by longitude, latitude, and depth. 
Within each stratum, commercial ages are weighted by the catch weight (kg) of the species in tows that were sampled, and survey ages are weighted by the catch density (kg/km$^2$) of the species in sampled tows. 
A second weighting is then applied: quarterly commercial ages are weighted by the commercial catch weight of the species from all tows within each quarter; stratum survey ages are weighted by stratum areas (km$^2$) in the survey. 

Ideally, sampling effort would be proportional to the amount of the species caught, but this is not usually the case. 
Personnel can control the sampling effort on surveys more than that aboard commercial vessels, but the relative catch among strata over the course of a year or survey cannot be known with certainty until the events have occurred. 
Therefore, the stratified weighting scheme presented below attempts to adjust for unequal sampling effort among strata.

For simplicity herein, we illustrate the weighting of age frequencies~$x_a$, unless otherwise specified. 
The weighting occurs at two levels: $h$ (quarters for commercial ages, strata for survey ages) and $i$ (years if commercial, surveys in series if survey). 
Notation is summarised in Table~\ref{tab:wtdAges}.

\usefont{\encodingdefault}{\familydefault}{\seriesdefault}{\shapedefault}\small
\begin{longtable}[1]{l>{\raggedright\arraybackslash}p{0.85\textwidth} }
%\caption{Equations for weighting age frequencies or proportions for \fishname.\\(\bold{c})~= commercial, (\bold{s})~= survey}
\caption{Equations for weighting age frequencies or proportions for Arrowtooth Flounder.\\(\textbf{c})~= commercial, (\textbf{s})~= survey}
\label{tab:wtdAges} \\
\hline
Symbol & Description \\ %\tstrut \bstrut \\
\hline
%& \tstrut \textbf{Indices} \bstrut \\
& \textbf{Indices} \\
$\bM{a}$ & age class (1 to $A$, where $A$ is an accumulator age-class) \\
$\bM{d}$ & (\textbf{c}) trip IDs as sample units \\
& (\textbf{s}) sample IDs as sample units \\
$\bM{h}$ & (\textbf{c}) quarters (1 to 4), 91.5 days each \\
& (\textbf{s}) strata (area-depth combinations) \\
$\bM{i}$ & (\textbf{c}) calendar years (1977 to present) \\
& (\textbf{s}) survey IDs in survey series (e.g., QCS Synoptic) \\ %\bstrut \\
\hline
%& \tstrut \textbf{Data} \bstrut \\
& \textbf{Data} \\
$\bM{x_{adhi}}$ & observations-at-age $a$ for sample unit $d$ in \eor{quarter}{stratum} $h$ of \eor{year}{survey} $i$ \\
$\bM{x^\prime_{adhi}}$ & proportion-at-age $a$ for sample unit $d$ in \eor{quarter}{stratum} $h$ of \eor{year}{survey} $i$ \\
$\bM{C_{dhi}}$ & (\textbf{c}) commercial catch (kg) of a given species for sample unit $d$ in quarter $h$ of year $i$ \\
& (\textbf{s}) density (kg/km$^2$) of a given species for sample unit $d$ in stratum $h$ of survey $i$ \\
$\bM{C^\prime_{dhi}}$ & $C_{dhi}$ as a proportion of total \eor{catch}{density} $C_{hi} = \sum_{d} C_{dhi}$ \\
$\bM{y_{ahi}}$ & weighted age frequencies at age $a$ in \eor{quarter}{stratum} $h$ of \eor{year}{survey} $i$ \\
$\bM{K_{hi}}$ & (\textbf{c}) total commercial catch (kg) of species in quarter $h$ of year $i$ \\
& (\textbf{s}) stratum area (km$^2$) of stratum $h$ in survey $i$ \\
$\bM{K_{hi}^\prime}$ & $K_{hi}$ as a proportion of total \eor{catch}{area} $K_i = \sum_{h} K_{hi}$ \\
$\bM{p_{ai}}$ & weighted frequencies at age $a$ in \eor{year}{survey} $i$ \\
$\bM{p_{ai}^\prime}$ & weighted proportions at age $a$ in \eor{year}{survey} $i$ \\ %\bstrut \\
\hline 
% \end{tabular}
% \end{center}
\end{longtable}
\usefont{\encodingdefault}{\familydefault}{\seriesdefault}{\shapedefault}\normalsize

For each \eor{quarter}{stratum} $h$ we weight sample unit frequencies $x_{ad}$ by sample unit \eor{catch}{density} of the assessment species. (For commercial ages, we use trip as the sample unit, though at times one trip may contain multiple samples. In these instances, multiple samples from a single trip will be merged into a single sample unit.) Within any \eor{quarter}{stratum} $h$ and \eor{year}{survey} $i$ there is a set of sample \eor{catches}{densities} $C_{dhi}$ that can be transformed into a set of proportions:
%
\eqn{C_{dhi}^\prime = \gfrac{C_{dhi}}{\sum_{d} C_{dhi}}~.}
%
The proportion $C_{dhi}^\prime$ is used to weight the age frequencies $x_{adhi}$ summed over $d$, which yields weighted age frequencies by \eor{quarter}{stratum} for each \eor{year}{survey}:
%
\eqn{y_{ahi} = \sum_{d} \big(C_{dhi}^\prime x_{adhi}\big)~.}
%
This transformation reduces the frequencies $x$ from the originals, and so we rescale (multiply) $y_{ahi}$ by the factor
%
\eqn{\gfrac{\sum_{a} x_{ahi}}{\sum_{a} y_{ahi}}}
%
to retain the original number of observations. (For proportions $x^\prime$ this is not needed.) Although we perform this step, it is strictly not necessary because at the end of the two-step weighting, we standardise the weighted frequencies to represent proportions-at-age.

At the second level of stratification by \eor{year}{survey} $i$, we calculate the the annual proportion of quarterly catch (t) for commercial ages or the survey proportion of stratum areas (km$^2$) for survey ages
%
\eqn{K_{hi}^\prime = \gfrac{K_{hi}}{\sum_{h} K_{hi}}}
%
to weight $y_{ahi}$ and derive weighted age frequencies by \eor{year}{survey}:
%
\eqn{p_{ai} = \sum_{h} \big(K_{hi}^\prime y_{ahi}\big)~.}
%
Again, if this transformation is applied to frequencies (as opposed to proportions), it reduces them from the original, and so we rescale (multiply) $p_{ai}$ by the factor
%
\eqn{\gfrac{\sum_{a} y_{ai}}{\sum_{a} p_{ai}}~.}
%
to retain the original number of observations.

Finally, we standardise the weighted frequencies to represent proportions-at-age:
%
\eqn{p_{ai}^\prime = \gfrac{p_{ai}}{\sum_{a} p_{ai}}~.}
%
If initially we had used proportions $x_{adhi}^\prime$ instead of frequencies $x_{adhi}$ , the final standardisation would not be necessary; however, its application does not affect the outcome.

The choice of data input (frequencies $x$ \emph{vs}. proportions $x^\prime$) can sometimes matter: the numeric outcome can be very different, especially if the input samples comprise few observations. Theoretically, weighting frequencies emphasises our belief in individual observations at specific ages while weighting proportions emphasises our belief in sampled age distributions. Neither method yields inherently better results; however, if the original sampling methodology favoured sampling few fish from many tows rather than sampling many fish from few tows, then weighting frequencies probably makes more sense than weighting proportions. In this assessment, we weight age frequencies $x$.

\newpage

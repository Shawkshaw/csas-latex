% CPUE.tex

\clearpage

\chapter{Standardisation of Commercial Trawl CPUE}

\section{Introduction}

Commercial catch and effort data have been used to generate indices of abundance in several ways. The simplest indices are derived from the arithmetic mean or geometric mean of catch divided by an appropriate measure of effort (Catch Per Unit Effort or CPUE) but such indices make no adjustments for changes in fishing practices or other non-abundance factors which may affect catch rates. Consequently, methods to standardise for changes to vessel configuration, the timing or location of catch and other possible effects have been developed to remove potential biases to CPUE that may result from such changes. In these models, abundance is represented as a“year effect”and the dependent variable is either an explicitly calculated CPUE represented as catch divided by effort, or an implicit CPUE represented as catch per tow or catch per record. In the latter case, additional effort terms can be offered as explanatory variables, allowing the model to select the effort term with the greatest explanatory power. It is always preferable to standardise for as many factors as possible when using CPUE as a proxy for abundance. Unfortunately, it is often not possible to adjust for factors that might affect the behaviour of fishers, particularly economic factors, resulting in indices that may not entirely reflect the underlying stock abundance.

\section{Methods}
\subsection{Arithmetic and Unstandardised CPUE}

Arithmetic and unstandardised CPUE indices provide potential measures of relative abundance, but are generally considered unreliable because they fail to take into account changes in the fishery, including spatial and temporal changes as well as behavioural and gear changes. They are frequently calculated because they provide a measure of the overall effect of the standardisation procedure.

Arithmetic CPUE $\left(\hat{A}_y\right)$ in year $y$ was calculated as the total catch for the year divided by the total effort in the year using equation~\ref{eq:cpueArith}:

\begin{align} \label{eq:cpueArith}
\hat{A}_y=\bM{\frac{\sum\limits_{i=1}^{n_y}{C_{i,y}}}{\sum\limits_{i=1}^{n_y}{E_{i,y}}}}
\end{align}
\begin{addmargin}[3em]{1em}
where $C_{i,y}$ is the catch, $E_{i,y}=T_{i,y}$ (tows) or $E_{i,y}=H_{i,y}$ (hours fished) for record $i$ in year $y$, and $n_y$ is the number of records in year $y$.
\end{addmargin}

Unstandardised (geometric) CPUE assumes a log-normal error distribution. An unstandardised index of CPUE $\left(\hat{G}_y\right)$ in year y was calculated as the geometric mean of the ratio of catch to effort for each record i in year y using  equation~\ref{eq:cpueGeom}:

\begin{align} \label{eq:cpueGeom}
\bM{\hat{G}_y=\bM{e^{\frac{\sum\limits_{i=1}^{n_y}{ln{\frac{C_{i,y}}{E_{i,y}}}}}{n_y}}}}
\end{align}
\begin{addmargin}[3em]{1em}
where $C_{i,y}$, $E_{i,y}$, and $n_y$ are as defined for equation~\ref{eq:cpueArith}.
\end{addmargin}

\subsection{Standardised CPUE}

These models are preferred over the unstandardised models described above because they can account for changes in fishing behaviour and other factors which may affect the estimated abundance trend, as long as the models are provided with adequate data. In the models described below, catch per record is used as the dependent variable and the associated effort is treated as an explanatory variable.

\subsubsection{Lognormal Model}

Standardised CPUE assumes a lognormal error distribution, with explanatory variables to used represent changes in the fishery. A standardised CPUE index (Eq.~\ref{eq:cpueLognormal}) is calculated from a generalised linear model (GLM) (\citet{qd99}) using a range of explanatory variables including [year], [month], [depth], [vessel] and other available factors:

\begin{align} \label{eq:cpueLognormal}
\bM{ln(I_i)=B+Y_{y_i}+\alpha_{a_i}+\beta_{b_i}+...+f(\chi_i)+f(\delta_i)+...+\varepsilon_i}
\end{align}
\begin{addmargin}[3em]{1em}
where $I_i=C_i$ (where: $B$ is the intercept; $Y_{y_i}$ is the year coefficient for the year corresponding to the $i$th record; $\alpha_{a_i}$ and $\beta_{b_i}$ are the coefficients for factorial variables $a$ and $b$ corresponding to the $i$th record; $f(\chi_i)$ and $f(\delta_i)$ are polynomial functions (to the 3\textsuperscript{rd} order) of the continuous variables $\chi_i$ and $\delta_i$ corresponding to the $i$th record; and $\varepsilon_i$ is an error term.
\end{addmargin}

The actual number of factorial and continuous explanatory variables in each model depends on the model selection criteria. Because each record represents a single tow, $C_i$ has an implicit associated effort of one tow. Hours fished for the tow is represented on the right-hand side of the equation, usually as a continuous (polynomial) variable.

Note that calculating standardised CPUE with Eq.~\ref{eq:cpueLognormal} without additional explanatory variables is equivalent to using Eq.~\ref{eq:cpueGeom}, provided the same definition for  is used.

Canonical coefficients and standard errors were calculated for each categorical variable (\citet{francis1999}). Standardised analyses typically set one of the coefficients to 1.0 without an error term and estimate the remaining coefficients and the associated error relative to the fixed coefficient. This is required because of parameter confounding. The \citet{francis1999} procedure rescales all coefficients so that the geometric mean of the coefficients is equal to 1.0 and calculates a standard error for each coefficient, including the fixed coefficient.

Coefficient-distribution-influence plots (CDI plots) are visual tools to facilitate understanding of patterns which may exist in the combination of coefficient values, distributional changes, and annual influence (\citet{bentley2011}). CDI plots were used to illustrate each explanatory variable added to the model.

\subsubsection{Binomial Logit Model}

The procedure described by Eq.~\ref{eq:cpueLognormal} is necessarily confined to the positive catch observations in the data set because the logarithm of zero is undefined. Observations with zero catch were modelled by fitting a logit regression model based on a binomial distribution and using the presence/absence of \fishname as the dependent variable (where 1 is substituted for $ln(I_i)$ in Eq.~\ref{eq:cpueLognormal} if it is a successful catch record and 0 if it is not successful) and using the same data set. Explanatory factors are estimated in the model in the same manner as described in Eq.~\ref{eq:cpueLognormal}. Such a model provides an alternative series of standardised coefficients of relative annual changes that is analogous to the series estimated from the lognormal regression.

\subsubsection{Combined Model}

A combined model, integrating the two sets of relative annual changes estimated by the lognormal and binomial models, can be estimated using the delta distribution, which allows zero and positive observations (\citet{vignaux1994}). Such a model provides a single index of abundance which integrates the signals from the positive (lognormal) and binomial series. This approach uses the following equation to calculate an index based on the two contributing indices:

\begin{align} \label{eq:cpueGeom}
\bM{^CY_y=\frac{^LY_y}{\left(1-P_0\left(1-\frac{1}{^BY_y}\right)\right)}}
\end{align}
\begin{addmargin}[3em]{1em}
where $^cY_y$ = combined index for year $y$, $^LY_y$ = lognormal index for year $I$, $^BY_y$ = binomial index for year $I$, and $P_0$ is the proportion zero for base year $0$. 
\end{addmargin}

\citet{francis2001} suggests that a bootstrap procedure is the appropriate way to estimate the variability of the combined index. Therefore, confidence bounds for the combined model were estimated using a bootstrap procedure based on 500 replicates, drawn with replacement.

\subsection{Preliminary inspection of the data}

Three separate but similar analyses are reported in this Appendix. Each are tow-by-tow analyses for combined DFO major areas using total catch (landings + discards) confined to the period 1996–2013, the period when detailed positional data for every tow are available and when there are estimates of discarded catch for the tow, because of the presence of an observer on board the vessel. These data are held in the DFO PacHarvestTrawl (PacHarv) and GFFOS databases (Fisheries and Oceans Canada, Pacific Region, Groundfish Data Unit);

Tow-by-tow catch and effort data for \fishname\ from the BC bottom trawl fishery operating off the west coast of Vancouver Island (WCVI: Areas 3C and 3D) or Queen Charlotte Strait (Areas 5A and 5B) or Hecate Strait and Dixon Entrance (Areas 5C and 5D) from 1996 to 2013 were selected using the following criteria:

\begin{itemize}
  \item{Tow start date between 1 January 1996 and 31 December 2013}
  \item{Bottom trawl type (includes soft and hard bottom trawl types after 2006) (includes ‘unknown’ gear)}
  \item{Fished in PMFC regions: 3C and 3D, or 5A and 5B, or 5C and 5D (depending on the analysis)}
  \item{Fishing success code $\leq$ 1 (code 0= unknown; code 1= useable)}
  \item{Catch of at least one fish or invertebrate species (no water hauls or inanimate object tows )}
  \item{Valid depth field}
  \item{Valid latitude and longitude co-ordinates}
  \item{Valid estimate of time towed that was greater than 0 hours and less than or equal to 6 hours}
\end{itemize}

Each record represents a single tow, which results in equivalency between the number of records and number of tows. Catch per record can therefore be used to represent CPUE, because each record (tow) has an implicit effort component. Table~\ref{tab:cpueTripsDepths} shows the empirical 1\% and 99\% quantiles of the distribution of depths for tows which captured \fishname\ in each of the three analyses. The bounding depths used in each analysis are also shown.

It is possible that the deeper recorded depths are in error or document tows that passed through a wide range of depths because the tow depth is taken from the depth recorded at the beginning of the tow (the field which is most commonly populated).

Vessels which only fished occasionally in each area or which did not actively fish \fishname\ were excluded from these analyses by setting minimum qualification criteria based on the number of trips per year and number of years fishing. A further restriction of stipulating a minimum number of successful tows was imposed in the 3CD and 5AB analyses. This was done to further reduce the number of qualifying vessels and to ensure that there were sufficient data to characterise the core fleet. This was not done for the 5CD analysis because of the small number of core vessels and the relatively large number of available qualifying tows for each vessel. Vessels which met the minimum criteria were included in the core vessel fleet, with all data for a qualifying vessel included, regardless of the number of trips in any year, once the vessel had been selected. Table~\ref{tab:cpueTripsQualify} gives the total number of vessels in each analysis, the number of qualifying trips per year, the number of years required and the minimum number of successful tows (if applicable). The final size of the core fleet is also given, as well as the proportion of the total fleet catch included in the analysis. Note that, in the 3CD analysis, one vessel was dropped because of uncertainties associated with the data. This vessel accounted for nearly 11,000 t of the total 60,700 t of catch in the dataset, with this amount of catch taken with just over 530 tows and about 500 hours of fishing. When this vessel was included in the analysis, its CPUE was approximately 60 times greater than the average for the rest of the fleet. The reason for this discrepancy is not known, but this vessel was dropped from the analysis, given its substantial difference from the remainder of the data. However, the removal of this vessel from the analysis had almost no effect on the estimated year indices.

Each of the figure references in table~\ref{tab:cpueTripsQualify} show the relationship of the two qualifying criteria with the resulting number of vessels and the proportion of catch retained in the analysis. The “bubble plot” references show the degree of vessel overlap across years for each analysis, with the requirement that the vessel data need to overlap in order for the model to correctly interpret the vessel and year categorical variables. These plots show that the overlap was good in all three models, with an adequate number of vessels operating across the 18 years of data. Only tows which were less than 6 hours long were used in the analyses. This criterion dropped very little data because 98\% to 99\% of the tows took less than 6 hours, depending on the analysis. The final data sets are large, ranging from about 30,000 to nearly 60,000 successful tows and 27,800 t to 40,000 t of total \fishname\ catch (Table~\ref{tab:cpueCatchRate}).

The explanatory variables found in table~\ref{tab:cpueExplanatory}  were offered to the model, based on the tow-by-tow information in each record.

\section{Results}

\subsection{3CD Tow-by-tow Analysis}

\subsubsection{Lognoirmal Positive Model}


\begin{table}[b]
%\tiny
\centering
\caption{\label{tab:cpueTripsDepths} Trips which qualify for the CPUE analyses. Vessels which did not actively fish \fishname\ were excluded by the criteria number of trips per year and number of years fishing.}
\begin{tabular}{lrrrrrl}
\hline
\specialcell{Analysis\\area} & 1\% (m) & 99\% (m) & \specialcell{Lower bound\\used} & \specialcell{Upper bound\\used} & \specialcell{\#depth\\bins} & Figure ref. \\
\hline
3CD & 71 & 750 & 60 & 760 & 18 & FIGREF \\
5AB & 64 & 439 & 40 & 460 &    & FIGREF \\
5CD & 39 & 388 & 20 & 400 & 10 & FIGREF \\
\hline
\end{tabular}
\end{table}


\begin{table}[b]
%\tiny
\centering
\caption{\label{tab:cpueTripsQualify} Trips which qualify for the CPUE analyses. Vessels which did not actively fish \fishname\ were excluded by the criteria number of trips per year and number of years fishing.}
\begin{tabular}{lrrrrrrll}
\hline
\specialcell{Analysis\\area} & \specialcell{Total\\vessels} & \specialcell{\#qualifying\\trips/year} & \specialcell{\#qualifying\\years} & \specialcell{Minimum \#\\tows} & \specialcell{\#qulifying\\vessels} & \specialcell{\% total\\catch} & Figure ref. & Bubble plot \\
\hline
3CD & 101 & 4 & 4 & 299 & 37 & 66\textsuperscript1 & FIGREF & FIGREF \\
5AB & 109 & 4 & 4 & 292 & 44 &                  79 & FIGREF & FIGREF \\
5CD &  91 & 5 & 5 &   0 & 18 &                  82 & FIGREF & FIGREF \\
\hline
\end{tabular}
\textsuperscript1After dropping the anomalous vessel described in the text
\end{table}

\begin{table}[b]
%\tiny
\centering
\caption{\label{tab:cpueCatchRate} Catch rates by area for \fishname\ }
\begin{tabular}{lrr|rr|lrr}
\hline
    &  &  & \multicolumn{2}{|r|}{Mean Catch Rate} \\
\specialcell{Analysis\\area} & \#tows & \specialcell{Total ARF\\catch (t)} & kg/tow & kg/h & \specialcell{Summary\\table ref.} & \specialcell{\#latitude\\bins} & \specialcell{\#locality\\bins} \\
\hline
3CD & 41,600 & 40,300 & 921 & 443 & TABREF & 22 & 27 \\
5AB & 57,900 & 33,200 & 534 & 250 & TABREF & 18 & 19 \\
5CD & 30,700 & 27,800 & 853 & 471 & TABREF & 22 & 21 \\
\hline
\end{tabular}
\end{table}

\begin{table}[b]
%\tiny
\centering
\caption{\label{tab:cpueExplanatory} Explanatory variables used in the model.}
\begin{tabular}{ll}
\hline
Year (1 January - 31 December) & 18 categories \\
\hline
Hours fished & continuous: 3rd order polynomial \\
\hline
Month & 12 categories\\
\hline
Latitude separated in 0.1° bands beginning with 48°N & \specialcell{variable for each analysis (see table~\ref{tab:cpueCatchRate})\\includes a final aggregated category} \\
\hline
DFO locality (Rutherford 1995) & \specialcell{variable for each analysis (see table~\ref{tab:cpueCatchRate})\\includes a final aggregated category} \\
\hline
Vessel & variable for each analysis (see table~\ref{tab:cpueCatchRate}) \\
\hline
Depth aggregated into 40 m depth bands & variable for each analysis (see table~\ref{tab:cpueCatchRate}) \\
\hline
DFO Major region & 2 categories for each analysis \\
\hline
\end{tabular}
\end{table}


\clearpage

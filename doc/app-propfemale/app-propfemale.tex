% appendix-propfemale.tex

\clearpage
\chapter{PROPORTION FEMALE ANALYSIS}
\label{chap:propfemale}

\section{INTRODUCTION}
This analysis was done to determine whether or not this stock has a sex proportion close to 50\% so that a sex-specific assessment model with an assumed sex proportion of 50\% could be used. If the proportion of females is large enough, a single-sex model could be employed. The weighting algorithm and proportion generation algorithm is similar to algorithm 2 in \citet{rocksole2013}. The analysis presented here assumes a coastwide stock.

\section{DATA SELECTION}
\textbf{Trawl Fishery}: \\
TRIP\_SUB\_TYPE\_DESC equal to 1 or 4 (observed domestic or non-observed domestic) \\
MORPHOMETRICS\_ATTRIBUTE\_CODE equal to 1 or 2 or 4 or 10 (Fork length, Standard length, Total length, Whole round weight) \\
\mbox{ }\\

\textbf{Surveys}: \\
TRIP\_SUB\_TYPE\_DESC equal to 2 or 3 (research or charter) \\
MORPHOMETRICS\_ATTRIBUTE\_CODE  equal to 1 or 2 or 4 or 10 (Fork length, Standard length, Total length, Whole round weight) \\
\mbox{ }\\

\textbf{Years}: \\
Greater than or equal to 1996 \\
\mbox{ }\\

\textbf{Quarters of the year}: \\
1 = Jan-Mar \\
2 = Apr-Jun \\
3 = Jul-Sep \\
4 = Oct-Dec \\
\mbox{ }\\

\textbf{Areas}: \\
3CD and 5ABCDE
\mbox{ }\\

\textbf{Sex}: \\
Male or female only, no unknowns or null fields \\
\mbox{ }\\

\section{ALGORITHM}

\subsection{Trawl Fishery}
Observations within a sample are likely to be correlated due to the small area which is trawled in a single fishing event. Also, trip samples
are likely to be correlated due to the fact that it is the same vessel and captain. This algorithm calculates a sex-specific mean weight by
trip, then combines the trips to get a mean weight within the quater of the year based on the total catch weight for the trip. It then combines
these across quarters by weighting the commercial catch in each quarter.

\subsection{Surveys}
For surveys, the same algorithm is followed except that the quarter of the year is not included in the calculation. This is because the surveys are
single events which occur linearly through a reletively short period of 1-2 months.

\subsection{Equations}

\begin{align} \label{eq:lw}
\hat{w}_{ijs}=\alpha_sl_{ijs}^{\beta_s}
\end{align}
\begin{addmargin}[3em]{1em}
where $\alpha_s$ and $\beta_s$ are parameters for sex $s$ and $w_{ijs}$ and $l_{ijs}$ are paired length-weight observations for specimen $i$ in sample $j$.
\end{addmargin}

\begin{align} \label{eq:samplewt}
W_{js}=\sum_{i=1}^{N_{js}}\hat{w}_{ijs}
\end{align}
\begin{addmargin}[3em]{1em}
where $W_{js}$ is the total weight for sample $j$, sex $s$, and $N_{js}$ is the number of specimens in sample $j$ for sex $s$
\end{addmargin}

\begin{align} \label{eq:totaltripwt}
W_{st}=\frac{\sum\limits_{j=1}^{K_t}W_{jst}S_{jt}}{\sum\limits_{j=1}^{K_t}S_{jt}}
\end{align}
\begin{addmargin}[3em]{1em}
where $W_{st}$ is the total weight for sex $s$ and trip $t$, $K_t$ is the number of samples in trip $t$, and $S_{jt}$ is the sample weight for sample $j$ from trip $t$.
\end{addmargin}

\begin{align} \label{eq:totaltripcatchwt}
C_t=\sum\limits_{j=1}^{K_t}C_{jt}
\end{align}
\begin{addmargin}[3em]{1em}
where $C_t$ is the total catch weight for sampled hauls for trip $t$, $K_j$ is the number of samples in trip $t$, and $C_{jt}$ is the catch weight associted with sample $j$ and trip $t$.
\end{addmargin}

\begin{align} \label{eq:totalquarterwt}
W_{qs}=\frac{\sum\limits_{t=1}^{T_q}W_{qst}R_{qt}}{\sum\limits_{n=1}^{T_q}R_{qt}}
\end{align}
\begin{addmargin}[3em]{1em}
where $W_{qs}$ is the total weight for sex $s$ and quarter of year $q$, $R_{qt}$ is the trip weight for all sampled trips in quarter $q$,
and $T_q$ is the number of sampled trips in quarter $q$.
\end{addmargin}

\begin{align} \label{eq:totalquartercatchwt}
C_q=\sum\limits_{t=1}^{K_q}C_{t}
\end{align}
\begin{addmargin}[3em]{1em}
where $C_q$ is the total catch weight for sampled hauls for quarter $q$, $K_q$ is the number of trips in quarter $q$, and $C_t$ is the catch weight associted with trip $t$.
\end{addmargin}

\begin{align} \label{eq:totalyearwt}
W_y=\frac{\sum\limits_{q=1}^{4}W_{qy}C_{qy}}{\sum\limits_{q=1}^{4}C_{qy}}
\end{align}
\begin{addmargin}[3em]{1em}
where $W_y$ is the total weight for year $y$, $W_{qy}$ is the weight in quarter $q$ of year $y$, and $C_{qy}$ is the catch in quarter $q$ of year $y$.
\end{addmargin}

\begin{align} \label{eq:totalyearcatchwt}
C_y=\sum\limits_{q=1}^{4}C_{qy}
\end{align}
\begin{addmargin}[3em]{1em}
where $C_y$ is the total catch weight for sampled hauls for year $y$, and $C_{qy}$ is the catch weight associted with quarter $q$ in year $y$.
\end{addmargin}

\begin{align} \label{eq:propfemale}
P_y=\frac{W_{s=Female,y}}{W_{s=Female,y}+W_{s=Male,y}}
\end{align}
\begin{addmargin}[3em]{1em}
where $P_y$ is the proportion female by weight for year $y$.
\end{addmargin}

\subsection{Pseudocode}
\clearpage
\begin{algorithm}[h]
\caption{Algortihm for calculating the proportion female}
\begin{algorithmic}[1]
  \Function{propfemale}{$()$}
    \Let{$i$}{Specimen}
    \Let{$s$}{Sex}
    \Let{$j$}{Sample number}
    \Let{$t$}{Trip number}
    \Let{$q$}{Quarter of year}
    \Let{$y$}{Year}
    \Let{$l_{ijs}$}{Specimen length measurement}
    \Let{$w_{ijs}$}{Specimen weight measurement}
    \Let{$\hat{w}_{ijs}$}{Specimen weight estimate}
    \For{each specimen $i$ where $w_{ijs}=NULL$ and $l_{ijs}<>NULL$}
      \LongState{apply the sex-specific length-weight relationship (Eq. \ref{eq:lw}) to fill in the missing specimen weights $w_{ijs}$ with estimates $\hat{w}_{ijs}$}
    \EndFor
    \For{each year $y$}
      \For{each quarter $q$ in year $y$}
        \For{each trip $t$ in quarter $q$}
          \For{each sample ID $j$ in trip $t$}
            \LongState{Calculate the sex-specific sample weight $W_{js}$ for sample $j$ (Eq. \ref{eq:samplewt})}
            \LongState{Extract the catch weight $C_j$ associated with sample $j$}
          \EndFor
          \LongState{Calculate the sex-specific total sample weight $W_{st}$ for trip $t$ (Eq. \ref{eq:totaltripwt})}
          \LongState{Calculate the sex-specific total catch weight $C_t$ for trip $t$ (Eq. \ref{eq:totaltripcatchwt})}
        \EndFor
        \LongState{Calculate the sex-specific total sample weight $W_{qs}$ for quarter $q$ (Eq. \ref{eq:totalquarterwt})}
        \LongState{Calculate the sex-specific total catch weight $C_q$ for quarter $q$ (Eq. \ref{eq:totalquartercatchwt})}
      \EndFor
      \LongState{Calculate the sex-specific total sample weight $W_{sy}$ for year $y$ weighted by catch $C_y$ (Eq. \ref{eq:totalyearwt})}
      \LongState{Calculate the proportion female for year $y$ (Eq. \ref{eq:propfemale})}
    \EndFor
  \EndFunction
\end{algorithmic}
\end{algorithm}

\subsection{Results}
The proportion of females resulting from this analysis are high, ranging from 0.714 for the 1998 HSMAS to 0.909 for the 2008 trawl fishery (Table \ref{tab:propfemale}).
\clearpage
\begin{table}[h]
\centering
\caption{\label{tab:propfemale} Proportion of females for the trawl fishery and 4 surveys}
\begin{tabular}{lccccc}
\hline \\
Year & Trawl Fishery & QCSSS & HSMSAS & HSSS & WCVISS \\
\hline \\
1996 & 0.854 & & & & \\
1997 & 0.901 & & & & \\
1998 & 0.778 & & 0.714 & & \\
1999 & 0.835 & & & & \\
2000 & 0.834 & & 0.908 & & \\
2001 & 0.879 & & & & \\
2002 & 0.857 & & 0.830 & & \\

2003 & 0.754 & 0.825 & & 0.838 & \\
2004 & 0.868 & 0.812 & & & 0.838 \\
2005 & 0.867 & 0.844 & & 0.771 & \\
2006 & 0.849 & & & & 0.897 \\
2007 & 0.839 & 0.764 & & 0.781 & \\
2008 & 0.909 & & & & 0.822 \\
2009 & 0.701 & 0.794 & & 0.763 & \\
2010 & 0.724 & & & & 0.813 \\
2011 & 0.739 & 0.748 & & 0.798 & \\
2012 & 0.827 & & & & 0.786 \\
2013 & 0.792 & 0.716 & & 0.753 & \\
2014 & 0.786 & & & & 0.777 \\
\hline \\
\end{tabular}
\end{table}

\newpage

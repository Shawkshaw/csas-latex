% preamble.tex - Tables of Contents, Figures, Tables, plus Abstract
%  Useful to comment out when don't need all that (e.g. working on an
%  Appendix).

%%CoverPage
%\input{./csasCoverPage} - currently using word->pdf and merging pdfs
%  for ResDocs.
%\newpage
% For working paper, use this before using Word for actual submission.

% From Matt Grinnel's Tech Report document.
%\setlength{\parindent}{0mm} 
\thispagestyle{fancyplain}
%\pagecolor{Tan}

\pagenumbering{roman}

\begin{flushleft}
\LARGE \textbf{Pacific Ocean Perch ({\bf \emph{Sebastes alutus}}) stock assessment for the west coast of Vancouver Island, British Columbia}

% \TRtitleCap}
\end{flushleft}
\vfill
{\Large Andrew M.~Edwards, Rowan Haigh and Paul J.~Starr}
\vfill
% \TRaddy
\vfill
% \TRyear
\vfill
{\LARGE \textbf{Working paper number 2012/P02a}}\\
 % Canadian Technical Report of\\Fisheries and Aquatic Sciences  \TRreportNumber}
\vspace{2cm}
[Replace with Word template for submission]
\vfill
% \lfoot{\includegraphics[height=5mm]{doc/DFOleft.jpeg}}
% \cfoot{}
% \rfoot{\includegraphics[height=5mm]{doc/DFOright.png}}
\clearpage





%% ---------------------------------------------------------------------


%%TOC will go here (page iii - req'mt by CSAS)
% \setcounter{page}{3}
\renewcommand{\contentsname}{\bf \large \vspace{-25mm} TABLE OF CONTENTS}
\addtocontents{toc}{\protect\thispagestyle{fancy}}
% \renewcommand{\cftchapterfont}{APPENDIX }\setlength{\cftfignumwidth}{1.5em}     % - ask Jaclyn, want to make it same as others.
%\begin{center}
%\tableofcontents
%\end{center}
%\newpage

% \nonumsection*{TABLE OF CONTENTS}

\begin{center}
\tableofcontents
\end{center}
\newpage


% Use Word template for these and cover page (use final YMR).


% Andy not using:
% \addtocontents{lof}{\protect\thispagestyle{fancy}}
%\renewcommand{\listfigurename}{\bf \large \vspace{-25mm} LIST OF MAIN FIGURES}     % JC and AME decided to restrict to just main ones.
%\renewcommand{\cftfigfont}{Figure }\setlength{\cftfignumwidth}{1.5em}
%\begin{center}
%\listoffigures
%\end{center}
%\addcontentsline{toc}{section}{LIST OF MAIN FIGURES}
%\newpage

%\addtocontents{lot}{\protect\thispagestyle{fancy}}
%\renewcommand{\listtablename}{\bf \large \vspace{-25mm} LIST OF MAIN TABLES}
%\renewcommand{\cfttabfont}{Table }\setlength{\cfttabnumwidth}{1.5em}
%\begin{center}
%\listoftables
%\end{center}
%\addcontentsline{toc}{section}{LIST OF MAIN TABLES}
%\clearpage

\leftskip=3em	%%required to indent Citation below
\parindent=-3em

{\bf Correct citation for this publication:}

Non-citable Working Paper.	%%(req'mt by CSAS)

%Cleary, J.S. and Haist, V. 2012.****
%Stock Assessment and Management Advice for the British Columbia Pacific Herring Stocks: 2012 Assessment and 2013 Forecasts.
%DFO Can. Sci. Advis. Sec. Res. Doc. 2012/xxx. xii + 151 p.

%JSC: would be nice to automate roman and regular page numbers. Counter? See Matt Grinnell's Tech Report latex files.

\leftskip=0em	%% end Citation indent
\parindent=-0em


%% \section*{Abstract}\addcontentsline{toc}{section}{ABSTRACT}
%% \nonumsection is centered
%\nonumsection*{ABSTRACT}\addcontentsline{toc}{section}{ABSTRACT}

% \newpage	%French abstract must appear on new page

%% \section*{R\'esum\'e}\addcontentsline{toc}{section}{R\'ESUM\'E}
% \nonumsection*{R\'ESUM\'E}\addcontentsline{toc}{section}{R\'ESUM\'E}

% \addcontentsline{toc}{section}{R\'ESUM\'E}


% hareng de la C.-B. sont g\'er\'es en fonction de cinq zones de stocks principales et de deux zones secondaires. En cons\'equence, l'information sur les prises et celle provenant des relev\'es est recueillie de fa\c{c}on ind\'ependante pour les sept zones, et on formule des avis scientifiques pour chacune. On s'est servi de toutes les donn\'ees biologiques disponibles sur la ponte, la composition selon l'\^{a}ge et la taille des stocks reproducteurs ainsi que sur les pr\'el\`{e}vements de la p\^{e}che commerciale pour d\'eterminer les niveaux d'abondance actuels. Ces derni\`{e}res ann\'ees, des examinateurs externes ont sugg\'er\'e que d'importantes r\'evisions soient apport\'ees au cadre d'\'evaluation du hareng, y compris au mod\`{e}le des prises selon l'\^{a}ge. Nous pr\'esentons donc un nouveau mod�le statistique int\'egr\'e des prises selon l'\^{a}ge (iSCAM) pour estimer conjointement l'abondance des stocks de hareng du Pacifique et les points de r\'ef\'erence connexes (partie I). Ce mod\`{e}le comprend des essais de simulation pour d\'emontrer que le mod\`ele peut estimer tous les param\`{e}tres ainsi qu'une param\'etrisation du nouveau mod\`{e}le d�\'evaluation comme le pr\'ec\'edent mod\`{e}le d�\'evaluation (mod\`{e}le des prises de hareng selon l'\^{a}ge ou HCAM) pour que l�on puisse comparer les estimations des param\`{e}tres et les estimations de la biomasse du stock reproducteur (en utilisant les donn\'ees de 1951 \`{a}) 2010) entre l'ancien mod\`{e}le (HCAM) et le nouveau (ISCAM). La partie II de ce document refl\`{e}te le nouveau cadre d'\'evaluation avec des donn\'ees des cinq zones de stock principales et des deux zones secondaires. Finalement, nous pr\'esentons des estimations de la biomasse avant la p\^{e}che et un avis sur les prises fond\'e sur les tableaux sur les d\'ecisions qui utilisent les pr\'evisions du recrutement \`{a} l'\^{a}ge 3 faible, moyen et bon ainsi que sur les tableaux sur la probabilit\'e du risque afin d'\'eclairer la prise de d\'ecision.



